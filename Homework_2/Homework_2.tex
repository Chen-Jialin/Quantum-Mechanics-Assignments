% !TEX program = pdflatex
% Quantum Mechanics Homework_2
\documentclass[12pt]{article}
\usepackage[margin=1in]{geometry} 
\usepackage{amsmath,amsthm,amssymb,amsfonts,enumitem,fancyhdr,color,comment,graphicx,environ}
\pagestyle{fancy}
\setlength{\headheight}{65pt}
\newenvironment{problem}[2][Problem]{\begin{trivlist}
\item[\hskip \labelsep {\bfseries #1}\hskip \labelsep {\bfseries #2.}]}{\end{trivlist}}
\newenvironment{sol}
    {\emph{Solution:}
    }
    {
    \qed
    }
\specialcomment{com}{ \color{blue} \textbf{Comment:} }{\color{black}} %for instructor comments while grading
\NewEnviron{probscore}{\marginpar{ \color{blue} \tiny Problem Score: \BODY \color{black} }}
\usepackage[UTF8]{ctex}
\lhead{Name: 陈稼霖\\ StudentID: 45875852}
\rhead{PHYS1501 \\ Quantum Mechanics \\ Semester Fall 2019 \\ Assignment 2}
\begin{document}
\begin{problem}{1}
Consider a particle in a complex potential $V(\vec{r})=U(\vec{r})+iW(r)$, where $U(\vec{r})$ and $W(\vec{r})$ are real functions.\\
(a) Derive the continuity equation for the time-dependent Schrödinger equation for a particle of mass m in the above complex potential.\\
(b) What is the integral form of the continuity equation?\\
(c) What is the condition on $W(\vec{r})$ for it to describe a source? What is the condition on $W(\vec{r})$ for it to describe a sink?
\end{problem}
\begin{sol}
(a) The Shrödinger equation
\begin{equation}
i\hbar\frac{\partial\psi(\vec{r},t)}{\partial t}=-\frac{\hbar^2}{2m}\nabla^2\psi(\vec{r},t)+V(\vec{r},t)\psi(\vec{r},t)=-\frac{\hbar^2}{2m}\nabla^2\psi(\vec{r},t)+[U(\vec{r},t)+iW(\vec{r},t)]\psi(\vec{r},t)
\end{equation}
Complex conjugate of the Shrödinger equation
\begin{equation}
-i\hbar\frac{\partial\psi^*(\vec{r},t)}{\partial t}=-\frac{\hbar^2}{2m}\nabla^2\psi^*(\vec{r},t)+[U(\vec{r},t)-iW(\vec{r},t)]\psi^*(\vec{r},t)
\end{equation}
Multiplying the Shrödinger equation with $\psi^*(\vec{r},t)$
\begin{equation}
i\hbar\psi^*\frac{\psi}{\partial t}=-\frac{\hbar^2}{2m}\psi^*\nabla^2\psi+(V+iW)\psi^*\psi
\end{equation}
Mutiplying the complex conjugate of the Shrödinger equation with $\psi(\vec{r},t)$
\begin{equation}
-i\hbar\psi\frac{\partial\psi^*}{\partial t}=-\frac{\hbar^2}{2m}\psi\nabla^2\psi^*+[V-iW]\psi\psi^*
\end{equation}
Substracting the two resultant equation
\begin{gather}
i\hbar(\psi^*\frac{\partial\psi}{\partial t}+\psi\frac{\partial\psi^*}{\partial t})=-\frac{\hbar^2}{2m}(\psi^*\nabla^2\psi-\psi\nabla^2\psi^*)+2iW\psi\psi^*\\
\Longrightarrow i\hbar\frac{\partial}{\partial t}(\psi^*\psi)=-\frac{\hbar^2}{2m}\nabla\cdot(\psi^*\nabla\psi-\psi\nabla\psi^*)+2iW\rho
\end{gather}
Defining the probability current density as
\begin{equation}
\vec{J}=\frac{\hbar}{2im}[\psi^*(\vec{r},t)\nabla\psi(\vec{r},t)-\psi(\vec{r},t)\nabla\psi^*(\vec{r},t)]
\end{equation}
Continuity equation
\begin{equation}
\frac{\partial\rho}{\partial t}+\nabla\vec{J}=\frac{2W}{\hbar}\rho
\end{equation}
(b) Integrating the continuity equation over $\Omega$
\begin{gather}
\int_{\Omega}\frac{\partial\rho}{\partial t}d^3r+\int_{\Omega}\nabla\cdot\vec{J}=\frac{2W}{\hbar}\int_{\Omega}\rho d^3r\\
\Longrightarrow\frac{\partial}{\partial t}\int_{\Omega}\rho d^3r+\int_{\Sigma}\vec{J}\cdot d\vec{S}=\frac{2W}{\hbar}\int_{\Omega}\rho d^3r
\end{gather}
(c) The continuity equation describes a source if $W(\vec{r})>0$.\\
The continuity equation describes a sink if $W(\vec{r})<0$.
\end{sol}

\begin{problem}{2}
Show that
\[
\hat{\vec{p}}^2=\frac{1}{r^2}\hat{\vec{L}}^2-\hbar^2\frac{1}{r^2}\frac{\partial}{\partial r}(r^2\frac{\partial}{\partial r})
\]
\end{problem}
\begin{sol}
Due to
\begin{equation}
\hat{\vec{p}}=-i\hbar\nabla
\end{equation}
left side of the equation
\begin{align}
\nonumber\hat{\vec{p}}^2=&-\hbar^2\nabla^2\\
=&-\hbar^2[\frac{1}{r^2\sin\theta}\frac{\partial}{\partial\theta}(\sin\theta\frac{\partial}{\partial\theta})+\frac{1}{r^2\sin^2\theta}\frac{\partial^2}{\partial\varphi^2}]-\hbar^2\frac{1}{r^2}\frac{\partial}{\partial r}(r^2\frac{\partial}{\partial r})
\end{align}
Due to
\begin{equation}
\hat{\vec{L}}=-i\hbar\vec{r}\times\nabla
\end{equation}
right side of the equation
\begin{align}
\nonumber&\frac{1}{r^2}\hat{\vec{L}}^2-\hbar^2\frac{1}{r^2}\frac{\partial}{\partial r}(r^2\frac{\partial}{\partial r})\\
\nonumber=&-\frac{\hbar^2}{r^2}(\vec{r}\times\nabla)\cdot(\vec{r}\times\nabla)-\hbar^2\frac{1}{r^2}\frac{\partial}{\partial r}(r^2\frac{\partial}{\partial r})\\
\nonumber=&-\hbar^2[\hat{r}\times(\frac{\partial}{\partial r}\hat{r}+\frac{1}{r}\frac{\partial}{\partial\theta}\hat{\theta}+\frac{1}{r\sin\theta}\frac{\partial}{\partial\varphi}\hat{\varphi})]\cdot[\hat{r}\times(\frac{\partial}{\partial r}\hat{r}+\frac{1}{r}\frac{\partial}{\partial\theta}\hat{\theta}+\frac{1}{r\sin\theta}\frac{\partial}{\partial\varphi}\hat{\varphi})]-\hbar^2\frac{1}{r^2}\frac{\partial}{\partial r}(r^2\frac{\partial}{\partial r})\\
\nonumber=&-\hbar^2(\frac{1}{r}\frac{\partial}{\partial\theta}\hat{\varphi}-\frac{1}{r\sin\theta}\frac{\partial}{\partial\varphi}\hat{\theta})\cdot(\frac{1}{r}\frac{\partial}{\partial\theta}\hat{\varphi}-\frac{1}{r\sin\theta}\frac{\partial}{\partial\varphi}\hat{\theta})-\hbar^2\frac{1}{r^2}\frac{\partial}{\partial r}(r^2\frac{\partial}{\partial r})\\
=&-\hbar^2[\frac{1}{r^2\sin\theta}\frac{\partial}{\partial\theta}(\sin\theta\frac{\partial}{\partial\theta})+\frac{1}{r^2\sin^2\theta}\frac{\partial^2}{\partial\varphi^2}]-\hbar^2\frac{1}{r^2}\frac{\partial}{\partial r}(r^2\frac{\partial}{\partial r})
\end{align}
Therefore
\begin{equation}
\hat{\vec{p}}^2=\frac{1}{r^2}\hat{\vec{L}}^2-\hbar^2\frac{1}{r^2}\frac{\partial}{\partial r}(r^2\frac{\partial}{\partial r})
\end{equation}
\end{sol}

\begin{problem}{3}
(a) Find the Taylor expansion of $\hat{f}(\lambda)=e^{\lambda\hat{A}}\hat{B}e^{-\lambda\hat{A}}$ with respect to $\lambda$ about $\lambda=0$. Here the operators $\hat{A}$ and $\hat{B}$ may not commute.\\
(b) Setting $\lambda=1$ in the above Taylor expansion of $\hat{f}=e^{\lambda\hat{A}}\hat{B}e^{-\lambda\hat{A}}$, derive an expansion for $e^{\hat{A}}\hat{B}e^{-\hat{A}}$.\\
(c) Using the expansion of $e^{\hat{A}}\hat{B}e^{-\hat{A}}$, evaluate $e^{i\hat{L}_y\theta/\hbar}\hat{L}_ze^{i\hat{L}_y\theta/\hbar}$.
\end{problem}
\begin{sol}
\\(a) The first-order derivative of $\hat{f}(\lambda)$ with respect to $\lambda$ about $\lambda=0$
\begin{equation}
\left.\frac{df}{d\lambda}\right|_{\lambda=0}=\left.\left(e^{\lambda\hat{A}}\hat{A}\hat{B}e^{-\lambda\hat{A}}-e^{\lambda\hat{A}}\hat{B}\hat{A}e^{-\lambda\hat{A}}\right)\right|_{\lambda=0}=\left.\left\{e^{\lambda\hat{A}}[\hat{A},\hat{B}]e^{-\lambda\hat{A}}\right\}\right|_{\lambda=0}=[\hat{A},\hat{B}]
\end{equation}
The second-order derivative of $\hat{f}(\lambda)$ with respect to $\lambda$ about $\lambda=0$
\begin{equation}
\left.\frac{d^2\hat{f}}{d\lambda^2}\right|_{\lambda=0}=\left.\left\{e^{\lambda\hat{A}}\hat{A}[\hat{A},\hat{B}]e^{-\lambda\hat{A}}-e^{\lambda\hat{A}}[\hat{A},\hat{B}]\hat{A}e^{-\lambda\hat{A}}\right\}\right|_{\lambda=0}=[\hat{A},[\hat{A},\hat{B}]]
\end{equation}
\[
\cdots\cdots
\]
%The $n$ th-order derivative of $\hat{f}(\lambda)$ with respect to $\lambda$ about $\lambda=0$
%\begin{equation}
%\hat{f}^{(n)}(0)=\sum_{k=0}^{n}(-1)^k\left(\begin{array}{c}n\\k\end{array}\right)\hat{A}^k\hat{B}\hat{A}^{n-k}
%\end{equation}
The Taylor expansion of $\hat{f}(\lambda)$ about $\lambda=0$
\begin{equation}
\hat{f}(\lambda)=\hat{B}+\frac{1}{1!}[\hat{A},\hat{B}]\lambda^1+\frac{1}{2!}[\hat{A},[\hat{A},\hat{B}]]\lambda^2+\frac{1}{3!}[\hat{A},[\hat{A},[\hat{A},\hat{B}]]]\lambda^3+\cdots
%\hat{f}(\lambda)=\sum_{n=0}^{\infty}\frac{\hat{f}^{(n)(0)}}{n!}\lambda^n=\sum_{n=0}^{\infty}\frac{\sum_{k=0}^n(-1)^k\left(\begin{array}{c}n\\k\end{array}\right)\hat{A}^k\hat{B}\hat{A}^{n-k}}{n!}\lambda^n
\end{equation}
(b) The $n$ th-order derivative of $\hat{f}(\lambda)$ with respect to $\lambda$ about $\lambda=1$
\begin{equation}
e^{\hat{A}}\hat{B}e^{-\hat{A}}=\hat{B}+\frac{1}{1!}[\hat{A},\hat{B}]\lambda^1+\frac{1}{2!}[\hat{A},[\hat{A},\hat{B}]]\lambda^2+\frac{1}{3!}[\hat{A},[\hat{A},[\hat{A},\hat{B}]]]\lambda^3+\cdots
%e^{\hat{A}}\hat{B}e^{-\hat{A}}=\hat{h}(1)=\sum_{n=0}^{\infty}\frac{\sum_{k=0}^n(-1)^k\left(\begin{array}{c}n\\k\end{array}\right)\hat{A}^k\hat{B}\hat{A}^{n-k}}{n!}
\end{equation}
(c)
\begin{equation}
e^{i\hat{L}_y\theta/\hbar}\hat{L}_ze^{i\hat{L}_y\theta/\hbar}=\hat{L}_z+\frac{i\theta}{1!\hbar}[\hat{L}_y,\hat{L}_z]+\frac{i^2\theta^2}{2!\hbar^2}[\hat{L}_y,[\hat{L}_y,\hat{L}_z]]+\frac{i^3\theta^3}{3!\hbar^3}[\hat{L}_y,[\hat{L}_y,[\hat{L}_y,\hat{L}_z]]]+\cdots
%e^{i\hat{L}_y\theta/\hbar}\hat{L}_ze^{i\hat{L}_y\theta/\hbar}=\sum_{n=0}^{\infty}\frac{\sum_{k=0}^n(-1)^k\left(\begin{array}{c}n\\k\end{array}\right)\left(\frac{i\hat{L}_y\theta}{\hbar}\right)^k\hat{L}_z\left(\frac{i\hat{L}_y\theta}{\hbar}\right)^{n-k}}{n!}
\end{equation}
Due to $[\hat{\lambda}_{\alpha},\hat{L}_{\beta}]=i\hbar\sum_{\gamma=x,y,z}\varepsilon_{\alpha\beta\gamma}\hat{L}_{\gamma}$
\begin{equation}
e^{i\hat{L}_y\theta/\hbar}\hat{L}_ze^{i\hat{L}_y\theta/\hbar}=\sum_{n=0}^{\infty}\frac{\theta^n}{n!}\left\{\begin{array}{ll}\hat{L}_z,&n\text{ mod }4=0\\\hat{L}_x,&n\text{ mod }4=1\\-\hat{L_z},&n\text{ mod }4=2\\-\hat{L}_x,&n\text{ mod }4=3\end{array}\right.
\end{equation}
Due to $\cos\theta=\sum_{k=0}^{\infty}\frac{(-1)^k\theta^{2k}}{(2k)!},\sin\theta=\sum_{k=0}^{\infty}\frac{(-1)^{2k+1}\theta^k}{(2k+1)!}$
\begin{equation}
e^{i\hat{L}_y\theta/\hbar}\hat{L}_ze^{i\hat{L}_y\theta/\hbar}=\hat{L}_z\cos\theta+\hat{L}_x\sin\theta
\end{equation}
\end{sol}

\begin{problem}{4}
The operators $\hat{A}$ and $\hat{B}$ do not commute, $[\hat{A},\hat{B}]=\hat{C}\neq0$, but they both commute with their commutator $\hat{C}$, $[\hat{A},\hat{C}]=[\hat{B},\hat{C}]=0$. Show that
\[
e^{\hat{A}+\hat{B}}=e^{\hat{A}}e^{\hat{B}}e^{-\hat{C}/2}=e^{\hat{B}}e^{\hat{A}}e^{\hat{C}/2}
\]
\end{problem}
\begin{sol}
Let
\begin{equation}
\hat{f}(\lambda)=e^{-\lambda\hat{B}}e^{-\lambda{A}}e^{\lambda(\hat{A}+\hat{B})}
\end{equation}
The derivative of $\hat{f}(\lambda)$
\begin{align}
\nonumber\frac{d\hat{f}}{d\lambda}=&-e^{-\lambda\hat{B}}\hat{B}e^{-\lambda\hat{A}}e^{\lambda(\hat{A}+\hat{B})}-e^{-\lambda\hat{B}}e^{-\lambda\hat{A}}\hat{A}e^{\lambda(\hat{A}+\hat{B})}+e^{-\lambda\hat{A}}e^{-\lambda\hat{B}}e^{\lambda(\hat{A}+\hat{B})}(\hat{A}+\hat{B})\\
\nonumber=&-e^{-\lambda\hat{B}}\hat{B}e^{-\lambda\hat{A}}e^{\lambda(\hat{A}+\hat{B})}+e^{-\lambda\hat{B}}e^{-\lambda\hat{A}}\hat{B}e^{\lambda(\hat{A}+\hat{B})}\\
=&-\hat{B}\hat{f}(\lambda)-e^{-\lambda\hat{B}}(e^{-\lambda\hat{A}}\hat{B}e^{\lambda\hat{A}})e^{-\lambda\hat{A}}e^{\lambda(\hat{A}+\hat{B})}
\end{align}
where
\begin{equation}
e^{-\lambda\hat{A}}\hat{B}e^{\lambda\hat{A}}=\hat{B}-\frac{1}{1!}[\hat{A},\hat{B}]\lambda^1+\frac{1}{2!}[\hat{A},[\hat{A},\hat{B}]]\lambda^2-\frac{1}{3!}[\hat{A},[\hat{A},[\hat{A},\hat{B}]]]\lambda^3+\cdots
\end{equation}
Due to $[\hat{A},\hat{B}]=\hat{C},[\hat{A},\hat{C}]=[\hat{B},\hat{C}]=0$
\begin{equation}
e^{-\lambda\hat{A}}\hat{B}e^{\lambda\hat{A}}=\hat{B}-\hat{C}\lambda
\end{equation}
so
\begin{align}
\frac{d\hat{f}}{d\lambda}=&-\hat{B}\hat{f}(\lambda)+e^{-\lambda\hat{B}}(\hat{B}-\hat{C}\lambda)e^{-\lambda}\hat{A}e^{\lambda(\hat{A}+\lambda\hat{B})}\\
=&-\lambda\hat{C}\hat{f}(\lambda)
\end{align}
integrate to get
\begin{gather}
\hat{f}(\lambda)=e^{-\lambda^2\hat{C}/2}\\
\Longrightarrow e^{\hat{A}+\hat{B}}=e^{\hat{A}}e^{\hat{B}}e^{-\hat{C}/2}
\end{gather}
Similarly, let
\begin{gather}
\hat{g}(\lambda)=e^{-\hat{A}}e^{-\hat{B}}e^{\hat{A}+\hat{B}}\\
\Longrightarrow\frac{dg}{d\lambda}=-\hat{A}\hat{g}(\lambda)+e^{-\lambda\hat{A}}(e^{-\lambda\hat{B}}\hat{A}e^{\lambda\hat{B}})e^{-\lambda\hat{B}}e^{\lambda(\hat{A}+\hat{B})}=-\hat{A}\hat{g}(\lambda)+e^{-\lambda\hat{A}}(\hat{A}-\lambda\hat{C})e^{-\lambda\hat{B}}e^{\lambda(\hat{A}+\hat{B})}\\
\Longrightarrow\hat{g}(\lambda)=e^{\lambda^2\hat{C}/2}\\
\Longrightarrow e^{\hat{A}+\hat{B}}=e^{\hat{B}}e^{\hat{A}}e^{\hat{C}/2}
\end{gather}
\end{sol}

\begin{problem}{5}
Consider a particle of mass $m$ subject to a potential $V(x)=\lambda|x|^n$ with $\lambda$ a constant, $n\neq-2$, and $-\infty<x<\infty$. The energy of the particle is given by $E=\frac{p^2}{2m}+\lambda|x|^n$.\\
(a) Making use of $|p|\sim\Delta p$, $\Delta x\Delta p\sim\hbar$, and $|x|\sim\Delta x/2$, express $E$ in terms of $\Delta x$.\\
(b) To obtain the ground-state energy, minimize $E$ with respect to $\Delta x$. Find the value of $\Delta x$ in the ground state.\\
(c) What is the expression of the ground-state energy?
\end{problem}
\begin{sol}
(a) $E$ in terms of $\Delta x$
\begin{equation}
E=\frac{p^2}{2m}+\lambda|x|^n=\frac{\hbar^2}{2m\Delta x^2}+\lambda(\frac{\Delta x}{2})^n
\end{equation}
(b) The derivative of $E$ about $\Delta x$
\begin{equation}
\frac{d E}{d\Delta x}=-\frac{\hbar^2}{m\Delta x^3}+\frac{n\lambda}{2}(\frac{\Delta x}{2})^{n-1}
\end{equation}
Let
\begin{gather}
\frac{d E}{d\Delta x}=0\\
\Longrightarrow\Delta x=\left(\frac{2^n\hbar^2}{nm\lambda}\right)^{1/(n+2)}
\end{gather}
(c) The expression of the ground-state energy
\begin{equation}
E_0=\frac{\hbar^2}{2m}\left(\frac{2^n\hbar^2}{nm\lambda}\right)^{-2/(n+2)}+\lambda\left(\frac{2^n\hbar^2}{nm\lambda}\right)^{n/(n+2)}
\end{equation}
\end{sol}
\end{document}