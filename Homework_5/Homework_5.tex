% !TEX program = pdflatex
% Quantum Mechanics Homework_5
\documentclass[12pt,a4paper]{article}
\usepackage[margin=1in]{geometry} 
\usepackage{amsmath,amsthm,amssymb,amsfonts,enumitem,fancyhdr,color,comment,graphicx,environ}
\pagestyle{fancy}
\setlength{\headheight}{65pt}
\newenvironment{problem}[2][Problem]{\begin{trivlist}
\item[\hskip \labelsep {\bfseries #1}\hskip \labelsep {\bfseries #2.}]}{\end{trivlist}}
\newenvironment{sol}
    {\emph{Solution:}
    }
    {
    \qed
    }
\specialcomment{com}{ \color{blue} \textbf{Comment:} }{\color{black}} %for instructor comments while grading
\NewEnviron{probscore}{\marginpar{ \color{blue} \tiny Problem Score: \BODY \color{black} }}
\usepackage[UTF8]{ctex}
\lhead{Name: 陈稼霖\\ StudentID: 45875852}
\rhead{PHYS1501 \\ Quantum Mechanics \\ Semester Fall 2019 \\ Assignment 5}
\begin{document}
\begin{problem}{1}
[C-T Exercise 2-1] $|\varphi_n\rangle$ are the eigenstates of a Hermitian operator $\hat{H}$ ($\hat{H}$ is, for example, the Hamiltonian of an arbitrary physical system). Assume that the states $|\varphi_n\rangle$ form a discrete orthonormal basis. The operator $\hat{U}(m,n)$ is defined by $\hat{U}(m,n)=|\varphi_m\rangle\langle\varphi_n|$.
\begin{itemize}
\item[(a)] Calculate the adjoint $\hat{U}^{\dagger}(m,n)$ of $\hat{U}(m,n)$.
\item[(b)] Calculate the commutator $[\hat{H},\hat{U}(m,n)]$.
\item[(c)] Prove the relation $\hat{U}(m,n)\hat{U}^{\dagger}(p,q)=\delta_{nq}\hat{U}(m,p)$.
\item[(d)] Calculate $\text{Tr}\{\hat{U}(m,n)\}$, the trace of the operator $\hat{U}(m,n)$.
\item[(e)] Let $\hat{A}$ be an operator, with matrix elements $A_{mn}=\langle\varphi_m|\hat{A}|\varphi_n\rangle$. Prove the relation $\hat{A}=\sum_{m,n}A_{mn}\hat{U}(m,n)$.
\item[(f)] Show that $A_{pq}=\text{Tr}\{\hat{A}U^{\dagger}(p,q)\}$.
\end{itemize}
\end{problem}
\begin{sol}
\begin{itemize}
\item[(a)]
\begin{align}
\nonumber&\langle\psi|U^{\dagger}(m,n)|\varphi\rangle=\langle\psi|(|\varphi_m\rangle\langle\varphi_n|)^{\dagger}|\varphi\rangle=[\langle\varphi|(|\varphi_m\rangle\langle\varphi_n|)|\psi\rangle]^*=[\langle\varphi|\varphi_m\rangle\langle\varphi_n|\psi\rangle]^*\\
=&\langle\varphi|\varphi_m\rangle^*\langle\varphi_n|\psi\rangle^*=\langle\varphi_m|\varphi\rangle\langle\psi|\varphi_n\rangle=\langle\psi|\varphi_n\rangle\langle\varphi_m|\varphi\rangle=\langle\psi|(|\varphi_n\rangle\langle\varphi_m|)|\varphi\rangle
\end{align}
Therefore,
\begin{equation}
\hat{U}^{\dagger}(m,n)=|\varphi_n\rangle\langle\varphi_m|
\end{equation}
\item[(b)]
\begin{align}
\nonumber[\hat{H},\hat{U}(m,n)]|\varphi\rangle=&\hat{H}\hat{U}(m,n)|\varphi\rangle-\hat{U}(m,n)\hat{H}|\varphi\rangle\\
\nonumber=&\hat{H}|\varphi_m\rangle\langle\varphi_n|\varphi\rangle-|\varphi_m\rangle\langle\varphi_n|\hat{H}|\varphi\rangle\\
\nonumber=&\hat{H}|\varphi_m\rangle\langle\varphi_n|\varphi\rangle-|\varphi_m\rangle\langle\varphi_n|\hat{H}^{\dagger}|\varphi\rangle\\
\nonumber=&H_m|\varphi_m\rangle\langle\varphi_n|\varphi\rangle-|\varphi_m\rangle\langle\varphi_n|H_n|\varphi\rangle\\
=&(H_m-H_n)|\varphi_m\rangle\langle\varphi_n|\varphi\rangle
\end{align}
Therefore,
\begin{equation}
[\hat{H},\hat{U}(m,n)]=(H_m-H_n)|\varphi_m\rangle\langle\varphi_n|
\end{equation}
\item[(c)]
\begin{align}
\nonumber\hat{U}(m,n)\hat{U}^{\dagger}(p,q)=&|\varphi_m\rangle\langle\varphi_n|(|\varphi_p\rangle\langle\varphi_q|)^{\dagger}\\
\nonumber=&|\varphi_m\rangle\langle\varphi_n|\varphi_q\rangle\langle\varphi_p|\\
\nonumber=&|\varphi_m\rangle\delta_{nq}\langle\varphi_p|\\
\nonumber=&\delta_{nq}|\varphi_m\rangle\langle\varphi_p|\\
=&\delta_{nq}\hat{U}(m,p)
\end{align}
\item[(d)]
\begin{align}
\nonumber\text{Tr}\{\hat{U}(m,n)\}=&\sum_i[\hat{U}(m,n)]_{ii}\\
\nonumber=&\sum_i\langle\varphi_i|\varphi_m\rangle\langle\varphi_n|\varphi_i\rangle\\
\nonumber=&\sum_i\delta_{im}\delta_{ni}\\
=&\delta_{mn}
\end{align}
\item[(e)]
\begin{align}
\nonumber\left[\sum_{m,n}A_{mn}\hat{U}(m,n)\right]_{ij}=&\langle\varphi_i|\left[\sum_{m,n}A_{mn}|\varphi_m\rangle\langle\varphi_n|\right]|\varphi_j\rangle\\
\nonumber=&\sum_{m,n}A_{mn}\langle\varphi_i|\varphi_m\rangle\langle\varphi_n|\varphi_j\rangle\\
\nonumber=&\sum_{m,n}A_{mn}\delta_{im}\delta_{nj}\\
=&A_{ij}=\langle\varphi_m|\hat{A}|\varphi_n\rangle
\end{align}
Therefore,
\begin{equation}
\hat{A}=\sum_{m,n}A_{mn}\hat{U}(m,n)
\end{equation}
\item[(f)]
\begin{align}
\nonumber\text{Tr}\{\hat{A}U^{\dagger}(p,q)\}=&\sum_i[\hat{A}U^{\dagger}(p,q)]_{ii}\\
\nonumber=&\sum_i\langle\varphi_i|\left[\left(\sum_{m,n}A_{mn}\hat{U}(m,n)\right)U^{\dagger}(p,q)\right]|\varphi_i\rangle\\
\nonumber=&\sum_i\langle\varphi_i|\left[\left(\sum_{m,n}A_{mn}|\varphi_m\rangle\langle\varphi_n|\right)U^{\dagger}(p,q)\right]|\varphi_i\rangle\\
\nonumber=&\sum_{mn}\left[A_{mn}\sum_i\langle\varphi_i|\varphi_m\rangle\langle\varphi_n|\varphi_q\rangle\langle\varphi_p|\varphi_i\rangle\right]\\
\nonumber=&\sum_{mn}\left[A_{mn}\sum_i\delta_{im}\delta_{nq}\delta_{pi}\right]\\
\nonumber=&\sum_{mn}A_{mn}\delta_{mp}\delta_{nq}\\
=&A_{pq}
\end{align}
\end{itemize}
\end{sol}

\begin{problem}{2}
[C-T Exercise 2-2] In a three-dimensional vector space, consider the operator whose matrix, in an orthonormal basis $\{|1\rangle,|2\rangle,|3\rangle\}$, is wirtten as $\hat{L}_y=\frac{\hbar}{\sqrt{2}}\left(\begin{array}{ccc}0&-i&0\\i&0&-i\\0&i&0\end{array}\right)$.
\begin{itemize}
\item[(a)] Is $\hat{L}_y$ Hermitian? Calculate its eigenvalues and eigenvectors (giving their normalized expansion in terms of the $\{|1\rangle,|2\rangle,|3\rangle\}$ basis).
\item[(b)]  Calculate the matrices which represent the projectors onto these eigenvectors. Then verify that they satisfy the orthogonality and closure relations.
\end{itemize}
\end{problem}
\begin{sol}
\begin{itemize}
\item[(a)] The Hermitian conjugate of $\hat{L}_y$
\begin{equation}
\hat{L}_y^{\dagger}=(\hat{L}_y^T)^*=
\frac{\hbar}{\sqrt{2}}\left(\begin{array}{ccc}
0&-i&0\\
-i&0&i\\
0&-i&0\\
\end{array}\right)
=\hat{L}_y
\end{equation}
Therefore, $\hat{L}_y$ is Hermitian.\\
The secular equation of $\hat{L}_y$
\begin{equation}
\det|A-\lambda I|=
\left|\begin{array}{ccc}
-\lambda&-i&0\\
i&-\lambda&-i\\
0&i&-\lambda\\
\end{array}\right|
=-\lambda^3+2\lambda=0
\end{equation}
The eigenvalues of $\hat{L}_y$ are
\begin{equation}
\lambda_1=\sqrt{2},\quad\lambda_2=0,\quad\lambda_3=-\sqrt{2}
\end{equation}
The normalized eigenvectors of $\hat{L}_y$ are
\begin{gather}
\nonumber(A-\lambda_1I)|\varphi_1\rangle
=
\left(\begin{array}{ccc}
-\sqrt{2}&-i&0\\
i&-\sqrt{2}&-i\\
0&i&-\sqrt{2}\\
\end{array}\right)
\left(\begin{array}{c}
c_1\\
c_2\\
c_3\\
\end{array}\right)
=
\left(\begin{array}{c}
0\\
0\\
0\\
\end{array}\right)\\
\Longrightarrow|\varphi_1\rangle=
\left(\begin{array}{c}
c_1\\
c_2\\
c_3\\
\end{array}\right)
=
\left(\begin{array}{c}
\frac{1}{2}\\
\frac{\sqrt{2}}{2}i\\
-\frac{1}{2}\\
\end{array}\right)\\
\nonumber(A-\lambda_2I)|\varphi_2\rangle
=
\left(\begin{array}{ccc}
0&-i&0\\
i&0&-i\\
0&i&0\\
\end{array}\right)
\left(\begin{array}{c}
c_1\\
c_2\\
c_3\\
\end{array}\right)
=
\left(\begin{array}{c}
0\\
0\\
0\\
\end{array}\right)\\
\Longrightarrow|\varphi_2\rangle=
\left(\begin{array}{c}
c_1\\
c_2\\
c_3\\
\end{array}\right)
=
\left(\begin{array}{c}
\frac{\sqrt{2}}{2}\\
0\\
\frac{\sqrt{2}}{2}\\
\end{array}\right)\\
\nonumber(A-\lambda_3I)|\varphi_3\rangle
=
\left(\begin{array}{ccc}
\sqrt{2}&-i&0\\
i&\sqrt{2}&-i\\
0&i&\sqrt{2}\\
\end{array}\right)
\left(\begin{array}{c}
c_1\\
c_2\\
c_3\\
\end{array}\right)
=
\left(\begin{array}{c}
0\\
0\\
0\\
\end{array}\right)\\
\Longrightarrow|\varphi_3\rangle=
\left(\begin{array}{c}
c_1\\
c_2\\
c_3\\
\end{array}\right)
=
\left(\begin{array}{c}
\frac{1}{2}\\
-\frac{\sqrt{2}}{2}i\\
-\frac{1}{2}\\
\end{array}\right)
\end{gather}
Rewritein them in terms of the $\{|1\rangle,|2\rangle,|3\rangle\}$ basis
\begin{gather}
|\varphi_1\rangle=\frac{1}{2}|1\rangle+\frac{\sqrt{2}}{2}i|2\rangle-\frac{1}{2}|3\rangle\\
|\varphi_2\rangle=\frac{\sqrt{2}}{2}|1\rangle+\frac{\sqrt{2}}{2}|3\rangle\\
|\varphi_3\rangle=\frac{1}{2}|1\rangle-\frac{\sqrt{2}}{2}i|2\rangle-\frac{1}{2}|3\rangle
\end{gather}
\item[(b)] The matrices which represent the projectors onto these eigenvectors
\begin{gather}
P_1=|\varphi_1\rangle\langle\varphi_1|=
\left(\begin{array}{c}
\frac{1}{2}\\
\frac{\sqrt{2}}{2}i\\
-\frac{1}{2}\\
\end{array}\right)
(\frac{1}{2}\quad\frac{\sqrt{2}}{2}i\quad-\frac{1}{2})
=
\left(\begin{array}{ccc}
\frac{1}{4}&\frac{\sqrt{2}}{4}i&-\frac{1}{4}\\
\frac{\sqrt{2}}{4}i&-\frac{1}{2}&-\frac{\sqrt{2}}{4}i\\
-\frac{1}{4}&-\frac{\sqrt{2}}{4}i&\frac{1}{4}\\
\end{array}\right)\\
P_2=|\varphi_2\rangle\langle\varphi_2|=
\left(\begin{array}{c}
\frac{\sqrt{2}}{2}\\
0\\
\frac{\sqrt{2}}{2}\\
\end{array}\right)
(\frac{\sqrt{2}}{2}\quad0\quad\frac{\sqrt{2}}{2})
=
\left(\begin{array}{ccc}
\frac{1}{2}&0&\frac{1}{2}\\
0&0&0\\
\frac{1}{2}&0&\frac{1}{2}\\
\end{array}\right)\\
P_3=|\varphi_3\rangle\langle\varphi_3|=
\left(\begin{array}{c}
\frac{1}{2}\\
-\frac{\sqrt{2}}{2}i\\
-\frac{1}{2}\\
\end{array}\right)
(\frac{1}{2}\quad-\frac{\sqrt{2}}{2}i\quad-\frac{1}{2})
=
\left(\begin{array}{ccc}
\frac{1}{4}&-\frac{\sqrt{2}}{4}i&-\frac{1}{4}\\
-\frac{\sqrt{2}}{4}i&-\frac{1}{2}&\frac{\sqrt{2}}{4}i\\
-\frac{1}{4}&\frac{\sqrt{2}}{4}i&\frac{1}{4}\\
\end{array}\right)
\end{gather}
\begin{gather}
P_1P_2=
\left(\begin{array}{ccc}
\frac{1}{4}&\frac{\sqrt{2}}{4}i&-\frac{1}{4}\\
\frac{\sqrt{2}}{4}i&-\frac{1}{2}&-\frac{\sqrt{2}}{4}i\\
-\frac{1}{4}&-\frac{\sqrt{2}}{4}i&\frac{1}{4}\\
\end{array}\right)
\left(\begin{array}{ccc}
\frac{1}{2}&0&\frac{1}{2}\\
0&0&0\\
\frac{1}{2}&0&\frac{1}{2}\\
\end{array}\right)
=
\left(\begin{array}{ccc}
0&0&0\\
0&0&0\\
0&0&0\\
\end{array}\right)\\
P_2P_3=
\left(\begin{array}{ccc}
\frac{1}{2}&0&\frac{1}{2}\\
0&0&0\\
\frac{1}{2}&0&\frac{1}{2}\\
\end{array}\right)
\left(\begin{array}{ccc}
\frac{1}{4}&-\frac{\sqrt{2}}{4}i&-\frac{1}{4}\\
-\frac{\sqrt{2}}{4}i&-\frac{1}{2}&\frac{\sqrt{2}}{4}i\\
-\frac{1}{4}&\frac{\sqrt{2}}{4}i&\frac{1}{4}\\
\end{array}\right)
=
\left(\begin{array}{ccc}
0&0&0\\
0&0&0\\
0&0&0\\
\end{array}\right)\\
P_3P_1=
\left(\begin{array}{ccc}
\frac{1}{4}&-\frac{\sqrt{2}}{4}i&-\frac{1}{4}\\
-\frac{\sqrt{2}}{4}i&-\frac{1}{2}&\frac{\sqrt{2}}{4}i\\
-\frac{1}{4}&\frac{\sqrt{2}}{4}i&\frac{1}{4}\\
\end{array}\right)
\left(\begin{array}{ccc}
\frac{1}{4}&\frac{\sqrt{2}}{4}i&-\frac{1}{4}\\
\frac{\sqrt{2}}{4}i&-\frac{1}{2}&-\frac{\sqrt{2}}{4}i\\
-\frac{1}{4}&-\frac{\sqrt{2}}{4}i&\frac{1}{4}\\
\end{array}\right)
=
\left(\begin{array}{ccc}
0&0&0\\
0&0&0\\
0&0&0\\
\end{array}\right)
\end{gather}
Therefore, the matrices which represent the projectors onto these eigenvectors satisfy the orthogonality relation.
\begin{align}
\nonumber P_1+P_2+P_3=&
\left(\begin{array}{ccc}
\frac{1}{4}&\frac{\sqrt{2}}{4}i&-\frac{1}{4}\\
\frac{\sqrt{2}}{4}i&-\frac{1}{2}&-\frac{\sqrt{2}}{4}i\\
-\frac{1}{4}&-\frac{\sqrt{2}}{4}i&\frac{1}{4}\\
\end{array}\right)
+
\left(\begin{array}{ccc}
\frac{1}{2}&0&\frac{1}{2}\\
0&0&0\\
\frac{1}{2}&0&\frac{1}{2}\\
\end{array}\right)
+
\left(\begin{array}{ccc}
\frac{1}{4}&-\frac{\sqrt{2}}{4}i&-\frac{1}{4}\\
-\frac{\sqrt{2}}{4}i&-\frac{1}{2}&\frac{\sqrt{2}}{4}i\\
-\frac{1}{4}&\frac{\sqrt{2}}{4}i&\frac{1}{4}\\
\end{array}\right)\\
=&\left(\begin{array}{ccc}
1&0&0\\
0&1&0\\
0&0&1\\
\end{array}\right)
\end{align}
\end{itemize}
\end{sol}

\begin{problem}{3}
[C-T Exercise 2-3] The state space of a certain physical system is three-dimensional. Let $\{|u_1\rangle,|u_2\rangle,|u_3\rangle\}$ be an orthonormal basis of this space. The kets $|\psi_0\rangle$ and $|\psi_1\rangle$ are confined by
\begin{gather*}
|\psi_0\rangle=\frac{1}{\sqrt{2}}|u_1\rangle+\frac{i}{2}|u_2\rangle+\frac{1}{2}|u_3\rangle\\
|\psi_1\rangle=\frac{1}{\sqrt{3}}|u_1\rangle+\frac{i}{\sqrt{3}}|u_3\rangle
\end{gather*}
\begin{itemize}
\item[(a)] Are these kets normalized?
\item[(b)] Calculate the matrices $\rho_0$ and $\rho_1$ representing, in the $\{|u_1\rangle,|u_2\rangle,|u_3\rangle\}$ basis, the projection operators onto the state $|\psi_0\rangle$ and onto the state $|\psi_1\rangle$. Verify that these matrices are Hermitian.
\end{itemize}
\end{problem}
\begin{sol}
\begin{itemize}
\item[(a)]
\begin{gather}
\begin{align}
\nonumber\langle\psi_0|\psi_0\rangle=&(\frac{1}{\sqrt{2}}\langle u_1|-\frac{i}{2}\langle u_2|+\frac{1}{2}\langle u_3|)(\frac{1}{\sqrt{2}}|u_1\rangle+\frac{i}{2}|u_2\rangle+\frac{1}{2}|u_3\rangle)\\
=&\frac{1}{2}\langle u_1|u_1\rangle+\frac{1}{4}\langle u_2|u_2\rangle+\frac{1}{4}\langle u_3|u_3\rangle=\frac{1}{2}+\frac{1}{4}+\frac{1}{4}=1
\end{align}\\
\begin{align}
\nonumber\langle\psi_1|\psi_1\rangle=&(\frac{1}{\sqrt{3}}\langle u_1|-\frac{i}{\sqrt{3}}\langle u_3|)(\frac{1}{\sqrt{3}}|u_1\rangle+\frac{i}{\sqrt{3}}|u_3\rangle)\\
=&\frac{1}{3}\langle u_1|u_1\rangle+\frac{1}{3}\langle u_3|u_3\rangle=\frac{1}{3}+\frac{1}{3}=\frac{2}{3}
\end{align}
\end{gather}
Therefore, ket $|\psi_0\rangle$ is normalized while ket $|\psi_1\rangle$ is not normalized.
\item[(b)] In the $\{|u_1\rangle,|u_2\rangle,|u_3\rangle\}$ basis, (where ket $|\psi_1\rangle$ got normalized)
\begin{gather}
|\psi_0\rangle=
\left(\begin{array}{c}
\frac{1}{\sqrt{2}}\\
\frac{i}{2}\\
\frac{1}{2}
\end{array}\right)\\
|\psi_1\rangle=
\left(\begin{array}{c}
\frac{1}{\sqrt{2}}\\
0\\
\frac{i}{\sqrt{2}}\\
\end{array}\right)
\end{gather}
The projector matrices onto the state $|\psi_0\rangle$ and onto the $|\psi_1\rangle$ are
\begin{gather}
\rho_0=|\psi_0\rangle\langle\psi_0|=
\left(\begin{array}{c}
\frac{1}{\sqrt{2}}\\
\frac{i}{2}\\
\frac{1}{2}
\end{array}\right)
(\frac{1}{\sqrt{2}}\quad-\frac{i}{2}\quad\frac{1}{2})
=
\left(\begin{array}{ccc}
\frac{1}{2}&-\frac{i}{2\sqrt{2}}&\frac{1}{2\sqrt{2}}\\
\frac{i}{2\sqrt{2}}&\frac{1}{4}&\frac{i}{4}\\
\frac{1}{2\sqrt{2}}&-\frac{i}{4}&\frac{1}{4}\\
\end{array}\right)\\
\rho_1=|\psi_1\rangle\langle\psi_1|=
\left(\begin{array}{c}
\frac{1}{\sqrt{2}}\\
0\\
\frac{i}{\sqrt{2}}
\end{array}\right)
(\frac{1}{\sqrt{2}}\quad0\quad-\frac{i}{\sqrt{2}})
=
\left(\begin{array}{ccc}
\frac{1}{2}&0&-\frac{i}{2}\\
0&0&0\\
\frac{i}{2}&0&\frac{1}{2}\\
\end{array}\right)
\end{gather}
\begin{gather}
\rho_0^{\dagger}=(\rho_0^T)^*=
\left(\begin{array}{ccc}
\frac{1}{2}&-\frac{i}{2\sqrt{2}}&\frac{1}{2\sqrt{2}}\\
\frac{i}{2\sqrt{2}}&\frac{1}{4}&\frac{i}{4}\\
\frac{1}{2\sqrt{2}}&-\frac{i}{4}&\frac{1}{4}\\
\end{array}\right)
=\rho_0\\
\rho_1^{\dagger}=(\rho_1^T)^*=
\left(\begin{array}{ccc}
\frac{1}{2}&0&-\frac{i}{2}\\
0&0&0\\
\frac{i}{2}&0&\frac{1}{2}\\
\end{array}\right)
=\rho_1
\end{gather}
Therefore, these matrices are Hermitian.
\end{itemize}
\end{sol}

\begin{problem}{4}
[C-T Exercise 2-9] Let $\hat{H}$  be the Hamiltonian operator of a physical system. Denote by $|\varphi_n\rangle$ the eigenvectors of $\hat{H}$, with eigenvalue $E_n$, $\hat{H}|\varphi_n\rangle=E_n|\varphi_n\rangle$.
\begin{itemize}
\item[(a)] For an arbitrary operator $\hat{A}$, prove the relation $\langle\varphi_n|[\hat{A},\hat{H}]|\varphi_n\rangle=0$.
\item[(b)] Consider a one-dimensional problem, where the physical system is a particle of mass $m$ and of potential energy $\hat{V}(\hat{x})$. In this case, $\hat{H}$ is written as $\hat{H}=\frac{1}{2m}\hat{p}^2+\hat{V}(\hat{x})$.
\begin{itemize}
\item[i.] In terms of $\hat{p},\hat{x}$, and $\hat{V}(\hat{x})$, find the commutators: $[\hat{H},\hat{p}],[\hat{H},\hat{x}]$ and $[\hat{H},\hat{x}\hat{p}]$.
\item[ii.] Show that the matrix element $\langle\varphi_n|\hat{p}|\varphi_n\rangle$ is zero.
\item[iii.] Establish a relation between $E_k=\langle\varphi_n|\frac{\hat{p}^2}{2m}|\varphi_n\rangle$ and $\langle\varphi_n|\hat{x}\frac{d\hat{V}(\hat{x})}{d\hat{x}}|\varphi_n\rangle$. Apply the derived relation to $\hat{V}(\hat{x})=V_0\hat{x}^{\lambda}$ with $\lambda=2,4,6,\cdots$ and $V_0>0$?
\end{itemize}
\end{itemize}
\end{problem}
\begin{sol}
\begin{itemize}
\item[(a)]
\begin{align}
\nonumber\langle\varphi_n|[\hat{A},\hat{H}]|\varphi_n\rangle=&\langle\varphi_n|[\hat{A}\hat{H}-\hat{H}\hat{A}]|\varphi_n\rangle\\
\nonumber=&\langle\varphi_n|\hat{A}\hat{H}|\varphi_n\rangle-\langle\varphi_n|\hat{H}\hat{A}|\varphi_n\rangle\\
\nonumber=&\langle\varphi_n|\hat{A}E_n|\varphi_n\rangle-\langle\varphi_n|E_n\hat{A}|\varphi_n\rangle\\
\nonumber=&E_n\langle\varphi_n|\hat{A}|\varphi_n\rangle-E_n\langle\varphi_n|\hat{A}|\varphi_n\rangle\\
=&0
\end{align}
\item[(b)]
\begin{itemize}
\item[i.]
\begin{align}
\nonumber[\hat{H},\hat{p}]=&\hat{H}\hat{p}-\hat{p}\hat{H}\\
\nonumber=&\left[\frac{1}{2m}\hat{p}^2+\hat{V}(\hat{x})\right]\hat{p}-\hat{p}\left[\frac{1}{2m}\hat{p}^2+\hat{V}(\hat{x})\right]\\
\nonumber=&\frac{1}{2m}\hat{p}^3+\hat{V}(\hat{x})\hat{p}-\frac{1}{2m}\hat{p}^3-\hat{p}\hat{V}(\hat{x})\\
=&\hat{V}(\hat{x})\hat{p}-\hat{p}\hat{V}(\hat{x})=[\hat{V}(\hat{x}),\hat{p}]
\end{align}
\begin{align}
\nonumber[\hat{H},\hat{x}]=&\hat{H}\hat{x}-\hat{x}\hat{H}\\
\nonumber=&\left[\frac{1}{2m}\hat{p}^2+\hat{V}(\hat{x})\right]\hat{x}-\hat{x}\left[\frac{1}{2m}\hat{p}^2+\hat{V}(\hat{x})\right]\\
\nonumber=&\frac{1}{2m}\hat{p}^2\hat{x}+\hat{V}(\hat{x})\hat{x}-\frac{1}{2m}\hat{x}\hat{p}^2-\hat{x}\hat{V}(\hat{x})\\
\nonumber=&\frac{1}{2m}(\hat{p}^2\hat{x}-\hat{x}\hat{p}^2)\\
\nonumber=&\frac{1}{2m}(\hat{p}^2\hat{x}-\hat{p}\hat{x}\hat{p}+\hat{p}\hat{x}\hat{p}-\hat{x}\hat{p}^2)\\
\nonumber=&\frac{1}{2m}[\hat{p}(\hat{p}\hat{x}-\hat{x}\hat{p})+(\hat{p}\hat{x}-\hat{x}\hat{p})\hat{p}]\\
\nonumber=&\frac{1}{2m}[\hat{p}[\hat{p},\hat{x}]+[\hat{p},\hat{x}]\hat{p}]\\
\nonumber=&\frac{1}{2m}[\hat{p}(-i\hbar)-i\hbar\hat{p}]\\
=&-\frac{i\hbar}{m}\hat{p}
\end{align}
\begin{align}
\nonumber[\hat{H},\hat{x}\hat{p}]=&\hat{H}\hat{x}\hat{p}-\hat{x}\hat{p}\hat{H}\\
\nonumber=&\left[\frac{1}{2m}\hat{p}^2+\hat{V}(\hat{x})\right]\hat{x}\hat{p}-\hat{x}\hat{p}\left[\frac{1}{2m}\hat{p}^2+\hat{V}(\hat{x})\right]\\
\nonumber=&\frac{1}{2m}\hat{p}^2\hat{x}\hat{p}+\hat{V}(\hat{x})\hat{x}\hat{p}-\frac{1}{2m}\hat{x}\hat{p}^3-\hat{x}\hat{p}\hat{V}(\hat{x})\\
\nonumber=&\frac{1}{2m}(\hat{p}^2\hat{x}-\hat{x}\hat{p}^2)\hat{p}+\hat{x}\hat{V}(\hat{x})\hat{p}-\hat{x}\hat{p}\hat{V}(\hat{x})\\
=&-\frac{i\hbar}{m}\hat{p}^2+\hat{x}[\hat{V}(\hat{x}),\hat{p}]
\end{align}
\item[ii.] According to the conclusion derived in (b) i., $\hat{p}=\frac{im}{\hbar}[\hat{H},\hat{x}]$, so
\begin{equation}
\langle\varphi_n|\hat{p}|\varphi_n\rangle=\frac{im}{\hbar}\langle\varphi_n|[\hat{H},\hat{x}]|\varphi_n\rangle=-\frac{im}{\hbar}\langle\varphi_n|[\hat{x},\hat{H}]|\varphi_n\rangle
\end{equation}
According to the conclusion derived in (a), $\langle\varphi_n|[\hat{A},\hat{H}]|\varphi_n\rangle=0$, therefore
\begin{equation}
\langle\varphi_n|\hat{p}|\varphi_n\rangle=-\frac{im}{\hbar}\langle\varphi_n|[\hat{x},\hat{H}]|\varphi_n\rangle=0
\end{equation}
\item[iii.] According to the conclusion derived in (b) i., $[\hat{H},\hat{x}\hat{p}]=-\frac{i\hbar}{m}\hat{p}^2+[\hat{V}(\hat{x}),\hat{x}\hat{p}]$, so
\begin{align}
\nonumber\langle\varphi_n|\frac{\hat{p}^2}{2m}|\varphi_n\rangle=&\langle\varphi_n|\left[-\frac{1}{2i\hbar}([\hat{H},\hat{x}\hat{p}]-\hat{x}[\hat{V}(\hat{x}),\hat{p}])\right]|\varphi_n\rangle\\
\nonumber=&-\frac{1}{2i\hbar}\left[\langle\varphi_n|[\hat{H},\hat{x}\hat{p}]|\varphi_n\rangle-\langle\varphi_n|\hat{x}[\hat{V}(\hat{x}),\hat{p}]|\varphi_n\rangle\right]\\
=&\frac{1}{2i\hbar}\langle\varphi_n|\hat{x}[\hat{V}(\hat{x}),\hat{p}]|\varphi_n\rangle
\end{align}
where
\begin{gather}
\begin{align}
\nonumber[\hat{V}(\hat{x}),\hat{p}]\psi(x)=&i\hbar[V(x)\frac{d}{dx}\psi(x)-\frac{d}{dx}V(x)\psi(x)]\\
\nonumber=&i\hbar[V(x)\frac{d}{dx}\psi(x)-\left(\frac{d}{dx}V(x)\right)\psi(x)-V(x)\frac{d}{dx}\psi(x)]\\
=&-i\hbar\left(\frac{d}{dx}V(x)\right)\psi(x)
\end{align}\\
\Longrightarrow[\hat{V}(\hat{x}),\hat{p}]=-i\hbar\left(\frac{d}{dx}V(x)\right)
\end{gather}
so
\begin{align}
\nonumber\langle\varphi_n|\frac{\hat{p}^2}{2m}|\varphi_n\rangle=&\frac{1}{2i\hbar}\langle\varphi_n|\hat{x}[\hat{V}(\hat{x}),\hat{p}]|\varphi_n\rangle\\
=&-\frac{1}{2}\langle\varphi_n|\hat{x}\frac{d\hat{V}(\hat{x})}{d\hat{x}}|\varphi_n\rangle
\end{align}
Apply the derived relation to $\hat{V}(\hat{x})=V_0\hat{x}^{\lambda}$ with $\lambda=2,4,6,\cdots$ and $V_0>0$
\begin{align}
\nonumber E_k=&-\frac{1}{2}\langle\varphi_n|\hat{x}\frac{\hat{V}(\hat{x})}{d\hat{x}}|\varphi_n\rangle\\
\nonumber=&-\frac{1}{2}\langle\varphi_n|\lambda V_0\hat{x}^{\lambda}|\varphi_n\rangle\\
\nonumber=&-\frac{\lambda}{2}V_0x^{\lambda}\\
=&-\frac{\lambda}{2}V(x)
\end{align}
\end{itemize}
\end{itemize}
\end{sol}

\begin{problem}{5}
[C-T Exercise 2-10] Using the relation $\langle x|p\rangle=(2\pi\hbar)^{-1/2}e^{ipx/\hbar}$, find the expression $\langle x|\hat{x}\hat{p}|\psi\rangle$ and $\langle x|\hat{p}\hat{x}|\psi\rangle$ in terms of $\psi(x)$. Can these results be found directly by using the fact that in the $\{|x\rangle\}$ representing, $\hat{p}$ acts like $-i\hbar\frac{d}{dx}$?
\end{problem}
\begin{sol}
\begin{gather}
\begin{align}
\nonumber\langle x|\hat{x}\hat{p}|\psi\rangle=&\langle x|x\hat{p}|\psi\rangle\\
\nonumber=&x\langle x|1\hat{p}1|\psi\rangle\\
\nonumber=&x\langle x|\int dp|p\rangle\langle p|\hat{p}\int dx|x\rangle\langle x|\psi\rangle\\
\nonumber=&x\langle x|\int dp|p\rangle\langle p|p\int dx|x\rangle\langle x|\psi\rangle\\
\nonumber=&x\int dp\langle x|p\rangle p\int dx\langle p|x\rangle\langle x|\psi\rangle\\
\nonumber=&x\int dp(2\pi\hbar)^{-1/2}e^{-ipx/\hbar}p\int dx(2\pi\hbar)^{-1/2}e^{ipx/\hbar}\psi(x)\\
\nonumber=&x\int dp(2\pi\hbar)^{-1/2}e^{-ipx/\hbar}p\bar{\psi}_p(x)\\
=&-i\hbar x\frac{d}{dx}\psi(x)
\end{align}\\
\begin{align}
\nonumber\langle x|\hat{p}\hat{x}|\psi\rangle=&\langle x|1\hat{p}\hat{x}|\psi\rangle\\
\nonumber=&\langle x|\int dp|p\rangle\langle p|\hat{p}\hat{x}|\psi\rangle\\
\nonumber=&\int dp\langle x|p\rangle\langle p|\hat{p}\hat{x}|\psi\rangle\\
\nonumber=&\int dp(2\pi\hbar)^{-1/2}e^{ipx/\hbar}\langle p|p\hat{x}|\psi\rangle\\
\nonumber=&\int dp(2\pi\hbar)^{-1/2}e^{ipx/\hbar}\langle p|p1\hat{x}|\psi\rangle\\
\nonumber=&\int dp(2\pi\hbar)^{-1/2}e^{ipx/\hbar}\langle p|p\int dx|x\rangle\langle x|\hat{x}|\psi\rangle\\
\nonumber=&\int dp(2\pi\hbar)^{-1/2}e^{ipx/\hbar}p\int dx\langle p|x\rangle\langle x|x|\psi\rangle\\
\nonumber=&\int dp(2\pi\hbar)^{-1/2}e^{ipx/\hbar}p\int dx(2\pi\hbar)^{-1/2}e^{-ipx/\hbar}x\psi(x)\\
\nonumber=&\int dp(2\pi\hbar)^{-1/2}e^{ipx/\hbar}pi\frac{d}{dp}\bar{\psi}_p(x)\\
\nonumber=&\int dp(2\pi\hbar)^{-1/2}e^{ipx/\hbar}i[\frac{d}{dp}(p\bar{\psi}_p(x))-\bar{\psi}_p(x)]\\
=&-i\hbar[x\frac{d}{dx}\psi(x)+\psi(x)]
\end{align}
\end{gather}
These results can be found directly by using the fact that in $|x\rangle$ representing, $\hat{p}$ act like $-i\hbar\frac{d}{dx}$:
\begin{gather}
\begin{align}
\nonumber\langle x|\hat{x}\hat{p}|\psi\rangle=&\int dx'\delta(x'-x)x'(-i\hbar\frac{d}{dx'})\psi(x')\\
\nonumber=&-i\hbar\int dx'\delta(x'-x)x'\frac{d}{dx'}\psi(x')\\
=&-i\hbar x\frac{d}{dx}\psi(x)
\end{align}\\
\begin{align}
\nonumber\langle x|\hat{p}\hat{x}|\psi\rangle=&\int dx'\delta(x'-x)(-i\hbar\frac{d}{dx'})x'\psi(x')\\
\nonumber=&-i\hbar\int dx'\delta(x'-x)[\psi(x')+x'\frac{d}{dx'}\psi(x')]\\
=&-i\hbar[\psi(x)+x\frac{d}{dx}\psi(x)]
\end{align}
\end{gather}
\end{sol}
\end{document}