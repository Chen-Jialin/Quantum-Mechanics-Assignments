% !TEX program = pdflatex
% Quantum Mechanics Homework_13
\documentclass[10pt,a4paper]{article}
\usepackage[margin=1in]{geometry} 
\usepackage{amsmath,amsthm,amssymb,amsfonts,enumitem,fancyhdr,color,comment,graphicx,environ}
\pagestyle{fancy}
\setlength{\headheight}{65pt}
\newenvironment{problem}[2][Problem]{\begin{trivlist}
\item[\hskip \labelsep {\bfseries #1}\hskip \labelsep {\bfseries #2.}]}{\end{trivlist}}
\newenvironment{sol}
    {\emph{Solution:}
    }
    {
    \qed
    }
\specialcomment{com}{ \color{blue} \textbf{Comment:} }{\color{black}}
\NewEnviron{probscore}{\marginpar{ \color{blue} \tiny Problem Score: \BODY \color{black} }}
\usepackage[UTF8]{ctex}
\usepackage{mathrsfs}
\lhead{Name: 陈稼霖\\ StudentID: 45875852}
\rhead{PHYS1501 \\ Quantum Mechanics \\ Semester Fall 2019 \\ Assignment 13}
\begin{document}
\begin{problem}{1}
[C-T Exercise 9-1] Consider a spin $1/2$ particle. Call its spin $\hat{\vec{S}}$, its orbital angular momentum $\hat{\vec{L}}$, and its state vector $|\psi\rangle$. The two functions $\psi_+(\vec{r})$ and $\psi_-(\vec{r})$ are defined by $\psi_{\pm}(\vec{r})=\langle\vec{r},\pm|\psi\rangle$. Assume that
\begin{gather}
\psi_+(\vec{r})=R(r)\left[Y_{00}(\theta,\phi)+\frac{1}{\sqrt{3}}Y_{10}(\theta,\phi)\right]\\
\psi_-(\vec{r})=\frac{R(r)}{\sqrt{3}}\left[Y_{11}(\theta,\phi)-Y_{10}(\theta,\phi)\right]
\end{gather}
where $r$, $\theta$, $\phi$ are the coordinates of the particle and $R(r)$ is a given function of $r$.
\begin{itemize}
\item[(a)] What condition must $R(r)$ satisfy for $|\psi\rangle$ to be normalized?
\item[(b)] $\hat{S}_z$ is measured with the particle in the state $|\psi\rangle$. What results can be found, and with what probabilities? Same question for $\hat{L}_z$, then for $\hat{S}_x$.
\item[(c)] A measurement of $\hat{\vec{L}}^2$, with the particle in the state $|\psi\rangle$, yielded zero. What state describes the particle just after this measurement? Same question if the measurement of $\hat{\vec{L}}^2$ had given $2\hbar^2$.
\end{itemize}
\end{problem}
\begin{sol}
\begin{itemize}
\item[(a)] The normalization condition is
\begin{align}
\nonumber\langle\psi|\psi\rangle=&\int d^3\vec{r}[|\psi_+(\vec{r})|^2+|\psi_-(\vec{r})|^2]\\
\nonumber=&\int_0^{+\infty}|R(r)|^2r^2dr\int_0^{\pi}\sin\theta d\theta\int_0^{2\pi}d\phi\\
\nonumber&\left[\left|Y_{00}(\theta,\phi)+\frac{1}{\sqrt{3}}Y_{10}(\theta,\phi)\right|^2+\left|\frac{1}{\sqrt{3}}Y_{11}(\theta,\phi)-\frac{1}{\sqrt{3}}Y_{10}(\theta,\phi)\right|^2\right]\\
=&2\int_0^{+\infty}|R(r)|^2r^2dr=1
\end{align}
Therefore, $R(r)$ must satisfy the condition that
\begin{equation}
\int_0^{\infty}|R(r)|^2r^2dr=\frac{1}{2}
\end{equation}
for $|\psi\rangle$ to be normalized.
\item[(b)] In the basis $\{|R,l,m,\varepsilon\rangle\}$, where $|R,l,m,\varepsilon\rangle=|R\rangle^{(r)}\otimes|l,m\rangle^{(\Omega)}\otimes|\varepsilon\rangle^{(s)}$, and $\langle R|R\rangle^{(r)}=\frac{1}{2}$, the state of the system can be written as
\begin{equation}
|\psi\rangle=|R,0,0,+\rangle+\frac{1}{\sqrt{3}}|R,1,0,+\rangle+\frac{1}{\sqrt{3}}|R,1,1,-\rangle-\frac{1}{\sqrt{3}}|R,1,0,-\rangle
\end{equation}
If $\hat{S}_z$ is measured with the particle in the state $|\psi\rangle$, the result $\hat{S}_z=\frac{\hbar}{2}$ can be found with probability
\begin{align}
\nonumber\mathscr{P}(\hat{S}_z=\frac{\hbar}{2})=&|\langle+|\psi\rangle|^2\\
\nonumber=&\left||R,0,0\rangle+\frac{1}{\sqrt{3}}|R,1,0\rangle\right|^2\\
\nonumber=&\frac{4}{3}\langle R|R\rangle\\
=&\frac{2}{3}
\end{align}
and the result $\hat{S}_z=-\frac{\hbar}{2}$ can be found with probability
\begin{equation}
\mathscr{P}(\hat{S}_z=-\frac{\hbar}{2})=1-\mathscr{P}(\hat{S}_z=\frac{\hbar}{2})=\frac{1}{3}
\end{equation}
If $\hat{L}_z$ is measured, the result $\hat{L}_z=\hbar$ can be found with probability
\begin{align}
\nonumber\mathscr{P}(\hat{L}_z=0)&|\langle m=1|\psi\rangle|^2\\
\nonumber=&\left|\frac{1}{\sqrt{3}}|R,1,-\rangle\right|^2\\
\nonumber=&\frac{1}{3}\langle R|R\rangle\\
=&\frac{1}{6}
\end{align}
and the result $\hat{L}_z=0$ can be found with probability
\begin{equation}
\mathscr{P}(\hat{L}_z=0)=1-\mathscr{P}(\hat{L}_z=\hbar)=\frac{5}{6}
\end{equation}
If $\hat{S}_x$ is measured, the result $\hat{S}_x=\frac{\hbar}{2}$ can be found with probability
\begin{align}
\nonumber\mathscr{P}(\hat{S}_x=\frac{\hbar}{2})=&|\langle\uparrow_x|\psi\rangle|^2\\
\nonumber=&\left|\frac{1}{\sqrt{2}}(\langle+|+\langle-|)|\psi\rangle\right|^2\\
\nonumber=&\frac{1}{2}\left||R,0,0\rangle+\frac{1}{\sqrt{3}}|R,1,1\rangle\right|^2\\
\nonumber=&\frac{2}{3}\langle R|R\rangle\\
=&\frac{1}{3}
\end{align}
the result $\hat{S}_x=\frac{\hbar}{2}$ can be found with probability
\begin{equation}
\mathscr{P}(\hat{S}_x=-\frac{\hbar}{2})=1-\mathscr{P}(\hat{S}_x=\frac{\hbar}{2})=\frac{2}{3}
\end{equation}
\item[(c)] The state describes the particle just after meansurement of $\hat{\vec{L}}^2$ yielding zero is
\begin{equation}
|\psi'\rangle=\frac{|l=0\rangle\langle l=0|\psi\rangle}{|\langle l=0|\psi\rangle|^2}=\sqrt{2}|R,0,0,+\rangle
\end{equation}
The state describes the particle just after measurement of $\hat{\vec{L}}^2$ yeilding $2\hbar^2$ is
\begin{equation}
|\psi'\rangle=\frac{|l=1\rangle\langle l=1|\psi\rangle}{|\langle l=1|\psi\rangle|^2}=\sqrt{\frac{2}{3}}(|R,1,0,+\rangle+|R,1,1,-\rangle-|R,1,0,-\rangle)
\end{equation}
\end{itemize}
\small{Reference: https://web.pa.msu.edu/people/mmoore/851HW13\_09Solutions.pdf}
\end{sol}

\begin{problem}{2}
[C-T Exercise 9-2] Consider a spin $1/2$ particle. $\hat{\vec{p}}$ and $\hat{\vec{S}}$ designate the observables  associated with its momentum and its spin. We choose as the basis of the state space the orthonormal basis $|p_xp_yp_z,\pm\rangle$ of eigenvectors common to $\hat{p}_x$, $\hat{p}_y$, $\hat{p}_z$, and $\hat{S}_z$ (whose eigenvalues are, respectively, $p_x$, $p_y$, $p_z$, and $\pm\hbar/2$). We intend to solve the eigenvalue equation of the operator $\hat{A}$ which is defined by $\hat{A}=\hat{\vec{S}}\cdot\hat{\vec{p}}$.
\begin{itemize}
\item[(a)] Is $\hat{A}$ Hermitian?
\item[(b)] Show that there exists a basis of eigenvectors of $\hat{A}$ which are also eigenvectors of $\hat{p}_x$, $\hat{p}_y$, and $\hat{p}_z$. In the subspace spanned by the kets $|p_xp_yp_z,\pm\rangle$, where $p_x$, $p_y$, and $p_z$ are fixed, what is the matrix representing $\hat{A}$?
\item[(c)] What are the eigenvalues of $\hat{A}$,  and what is their degree of degeneracy? Find a system of eigenvectors common to $\hat{A}$ and $\hat{p}_x$, $\hat{p}_y$, $\hat{p}_z$.
\end{itemize}
\end{problem}
\begin{sol}
\begin{itemize}
\item[(a)] The operator $\hat{A}$ can be expressed as
\begin{equation}
\hat{A}=(\hat{S}_x\vec{e}_x+\hat{S}_y\vec{e}_y+\hat{S}_z\vec{e}_z)\cdot(\hat{p}_x\vec{e}_x+\hat{p}_y\vec{e}_y+\hat{p}_z\vec{e}_z)=\hat{S}_x\hat{p}_x+\hat{S}_y\hat{p}_y+\hat{S}_z\hat{p}_z
\end{equation}
The Hermitian conjugate of the operator $\hat{A}$ is
\begin{equation}
\hat{A}^{\dagger}=\hat{p}_x^{\dagger}\hat{S}_x^{\dagger}+\hat{p}_y^{\dagger}\hat{S}_y^{\dagger}+\hat{p}_z^{\dagger}\hat{S}_z^{\dagger}=\hat{p}_x\hat{S}_x+\hat{p}_y\hat{S}_y+\hat{p}_z\hat{S}_z
\end{equation}
Since the momentum operator and the spin operator act on two different Hilbert spaces, they commute
\begin{gather}
[\hat{S}_x,\hat{p}_x]=\hat{S}_x\hat{p}_x-\hat{p}_x\hat{S}_x=0\\
[\hat{S}_y,\hat{p}_y]=\hat{S}_y\hat{p}_y-\hat{p}_y\hat{S}_y=0\\
[\hat{S}_z,\hat{p}_z]=\hat{S}_z\hat{p}_z-\hat{p}_z\hat{S}_z=0
\end{gather}
so the Hermitian conjugate of the operator $\hat{A}$ can be written as
\begin{equation}
\hat{A}^{\dagger}=\hat{S}_x\hat{p}_x+\hat{S}_y\hat{p}_y+\hat{S}_z\hat{p}_z=\hat{A}
\end{equation}
Therefore, $\hat{A}$ is Hermitian.
\item[(b)] 
\begin{align}
\nonumber\langle p_xp_yp_z,+|\hat{A}|p_xp_yp_z,+\rangle=&\langle p_xp_yp_z,+|(\hat{S}_x\hat{p}_x+\hat{S}_y\hat{p}_y+\hat{S}_z\hat{p}_z)|p_xp_yp_z,+\rangle\\
\nonumber=&\langle p_xp_yp_z,+|(\frac{\hat{S}_++\hat{S}_-}{2}\hat{p}_x+\frac{\hat{S}_+-\hat{S}_-}{2i}\hat{p}_y+\hat{S}_z\hat{p}_z)|p_xp_yp_z,+\rangle\\
\nonumber=&\langle p_xp_yp_z,+|(p_x\frac{\hat{S}_++\hat{S}_-}{2}+p_y\frac{\hat{S}_+-\hat{S}_-}{2i}+p_z\hat{S}_z)|p_xp_yp_z,+\rangle\\
=&\frac{\hbar}{2}p_z
\end{align}
\begin{align}
\nonumber\langle p_xp_yp_z,+|\hat{A}|p_xp_yp_z,-\rangle=&\langle p_xp_yp_z,+|(p_x\frac{\hat{S}_++\hat{S}_-}{2}+p_y\frac{\hat{S}_+-\hat{S}_-}{2i}+p_z\hat{S}_z)|p_xp_yp_z,-\rangle\\
=&\frac{\hbar}{2}(p_x-ip_y)
\end{align}
\begin{align}
\nonumber\langle p_xp_yp_z,-|\hat{A}|p_xp_yp_z,+\rangle=&\langle p_xp_yp_z,-|(p_x\frac{\hat{S}_++\hat{S}_-}{2}+p_y\frac{\hat{S}_+-\hat{S}_-}{2i}+p_z\hat{S}_z)|p_xp_yp_z,+\rangle\\
=&\frac{\hbar}{2}(p_x+ip_y)
\end{align}
\begin{align}
\nonumber\langle p_xp_yp_z,-|\hat{A}|p_xp_yp_z,-\rangle=&\langle p_xp_yp_z,-|(p_x\frac{\hat{S}_++\hat{S}_-}{2}+p_y\frac{\hat{S}_+-\hat{S}_-}{2i}+p_z\hat{S}_z)|p_xp_yp_z,-\rangle\\
=&-\frac{\hbar}{2}p_z
\end{align}
Therefore, in the subspace spanned by the kets $|p_xp_yp_z,\pm\rangle$, the matrix representing $\hat{A}$ is
\begin{equation}
\hat{A}=\frac{\hbar}{2}\left(\begin{array}{cc}
p_z&p_x-ip_y\\
p_x+ip_y&-p_z
\end{array}\right)
\end{equation}
The characteristic equation of $\hat{A}$ is
\begin{equation}
|\hat{A}-\lambda I|=\left|\begin{array}{cc}
\frac{\hbar}{2}p_z-\lambda&\frac{\hbar}{2}(p_x-ip_y)\\
\frac{\hbar}{2}(p_x+ip_y)&-\frac{\hbar}{2}p_z-\lambda
\end{array}\right|=\lambda^2-\frac{\hbar^2}{4}(p_x^2+p_y^2+p_z^2)=0
\end{equation}
The eigenvalues of $\hat{A}$ is
\begin{equation}
\lambda_{1,2}=\pm\frac{\hbar}{2}\sqrt{p_x^2+p_y^2+p_z^2}
\end{equation}
Since $\hat{A}$ has two different eigenvalues given $p_x$, $p_y$ and $p_z$ fixed, it has two linear-independent eigenvector in the basis of $\{|p_xp_yp_z,\pm\rangle\}$, which are also the eigenvectors of $\hat{p}_x$, $\hat{p}_y$ and $\hat{p}_z$. This means that any arbitrary ket $|p_xp_yp_z,\pm\rangle$ can be wirtten as one and only one linear combination of the eigenvectors of $\hat{A}$. Since $\{|p_xp_yp_z,\pm\rangle\}$ is a basis of the state space, any state can be written as one and only one combination of $|p_xp_yp_z,\pm\rangle$. In this way, any state can be written as one and only one combination of the eigenvectors of $\hat{A}$. Therefore, there exists a basis of eigenvectors of $\hat{A}$ which are also eigenvectors of $\hat{p}_x$, $\hat{p}_y$ and $\hat{p}_z$.
\item[(c)] As obtained in (b) above, the eigenvalues of $\hat{A}$ are
\begin{gather}
\lambda_1=\frac{\hbar}{2}\sqrt{p_x^2+p_y^2+p_z^2}\\
\lambda_2=-\frac{\hbar}{2}\sqrt{p_x^2+p_y^2+p_z^2}
\end{gather}
Their degeneracy is infinite since infinite sets of $\{p_x,p_y,p_z\}$ can make equal $\sqrt{p_x^2+p_y^2+p_z^2}$.\\
The eigenvector corresponding to $\lambda_1$ is
\begin{align}
\nonumber|\psi_1\rangle=&\left(\begin{array}{c}
\frac{p_x-ip_y}{\sqrt{2(p_x^2+p_z^2-ip_xp_y-p_z\sqrt{p_x^2+p_y^2+p_z^2})}}\\
\frac{-p_z+\sqrt{p_x^2+p_y^2+p_z^2}}{\sqrt{2(p_x^2+p_z^2-ip_xp_y-p_z\sqrt{p_x^2+p_y^2+p_z^2})}}
\end{array}\right)\\
=&\frac{(p_x-ip_y)|p_xp_yp_z,+\rangle+(p_z-\sqrt{p_x^2+p_y^2+p_z^2})|p_xp_yp_z,-\rangle}{\sqrt{2(p_x^2+p_z^2-ip_xp_y-p_z\sqrt{p_x^2+p_y^2+p_z^2})}}
\end{align}
The eigenvector corresponding to $\lambda_2$ is
\begin{align}
\nonumber|\psi_2\rangle=&\left(\begin{array}{c}
\frac{p_x-ip_y}{\sqrt{2(p_x^2+p_z^2-ip_xp_y+p_z\sqrt{p_x^2+p_y^2+p_z^2})}}\\
\frac{-p_z-\sqrt{p_x^2+p_y^2+p_z^2}}{\sqrt{2(p_x^2+p_z^2-ip_xp_y+p_z\sqrt{p_x^2+p_y^2+p_z^2})}}
\end{array}\right)\\
=&\frac{(p_x-ip_y)|p_xp_yp_z,+\rangle-(p_z+\sqrt{p_x^2+p_y^2+p_z^2})|p_xp_yp_z,-\rangle}{\sqrt{2(p_x^2+p_z^2-ip_xp_y+p_z\sqrt{p_x^2+p_y^2+p_z^2})}}
\end{align}
These eigenvectors are common to $\hat{A}$ and $\hat{p}_x$, $\hat{p}_y$, $\hat{p}_z$.
\end{itemize}
\end{sol}

\begin{problem}{3}
[C-T Exercise 9-3] The Hamiltonian of an electron of mass $m$, charge $q$, spin $\hbar\vec{\sigma}/2$ with $\sigma_x$, $\sigma_y$, and $\sigma_z$ the Pauli matrices, placed in an electromagnetic field described by the vector potential $\hat{A}(\vec{r},t)$ and the scalar potential $U(\vec{r},t)$, is written $\hat{H}=\frac{1}{2m}[\hat{\vec{p}}-q\vec{A}(\hat{\vec{r}},t)]^2+qU(\hat{\vec{r}},t)-\frac{q\hbar}{2m}\vec{\sigma}\cdot\vec{B}(\hat{\vec{r}},t)$. The last term represents the interaction between the spin magnetic moment $(q\hbar/2m)\vec{\sigma}$ and the magnetic field $\vec{B}(\hat{\vec{r}},t)=\vec{\nabla}\times\vec{A}(\hat{\vec{r}},t)$. Show, using the properties of the Pauli matrices, that this Hamiltonian can also be written in the form ("the Pauli Hamiltonian") $\hat{H}=\frac{1}{2m}\left\{\vec{\sigma}\cdot[\hat{\vec{p}}-q\vec{A}(\hat{\vec{r}},t)]\right\}^2+qU(\hat{\vec{r}},t)$.
\end{problem}
\begin{sol}
Let the Hamiltonian operates on a state function
\begin{align}
\nonumber\hat{H}\psi=&\frac{1}{2m}[\hat{\vec{p}}-q\vec{A}(\hat{\vec{r}},t)]^2\psi+qU(\hat{\vec{r}},t)\psi-\frac{q\hbar}{2m}\vec{\sigma}\cdot\vec{B}(\hat{\vec{r}},t)\\
\nonumber=&\frac{1}{2m}\{[\vec{p}-q\vec{A}(\hat{\vec{r}},t)]^2-iq\vec{\sigma}\cdot[-i\hbar\nabla\times\vec{A}(\hat{\vec{r}},t)]\}\psi+qU(\hat{\vec{r}},t)\psi\\
\nonumber=&\frac{1}{2m}\{[\vec{p}-q\vec{A}(\hat{\vec{r}},t)]^2\psi-iq\vec{\sigma}\cdot[\hat{\vec{p}}\times\vec{A}(\hat{\vec{r}},t)]\psi\}+qU(\hat{\vec{r}},t)\psi\\
\label{operate}=&\frac{1}{2m}\{[\vec{p}-q\vec{A}(\hat{\vec{r}},t)]^2\psi-i\vec{\sigma}\cdot[\hat{\vec{p}}\times q\vec{A}(\hat{\vec{r}},t)]\psi\}+qU(\hat{\vec{r}},t)\psi
\end{align}
where
\begin{align}
\nonumber[\hat{\vec{p}}\times q\hat{A}(\hat{\vec{r}},t)]\psi=&[\hat{\vec{p}}\times q\hat{A}(\hat{\vec{r}},t)]\psi+(\hat{\vec{p}}\psi)\times q\vec{A}(\hat{\vec{r}},r)-(\hat{\vec{p}}\psi)\times q\vec{A}(\hat{\vec{r}},r)
\end{align}
Using
\begin{equation}
\vec{A}\times\vec{B}\phi=(\vec{A}\times\vec{B})\phi+(\hat{A}\phi)\times\vec{B}
\end{equation}
we have
\begin{align}
\nonumber[\hat{\vec{p}}\times q\hat{A}(\hat{\vec{r}},t)]\psi=&\hat{\vec{p}}\times q\vec{A}(\hat{\vec{r}},t)\psi-(\hat{\vec{p}}\psi)\times q\vec{A}(\hat{\vec{r}},t)\\
\nonumber=&-[\hat{\vec{p}}\times\hat{\vec{p}}-\vec{p}\times q\vec{A}(\hat{\vec{r}},t)-q\vec{A}(\hat{\vec{r}},t)\times\hat{\vec{p}}+q\vec{A}(\hat{\vec{r}},t)\times q\vec{A}(\hat{\vec{r}},t)]\psi\\
=&-[\hat{\vec{p}}-q\vec{A}(\hat{\vec{r}},t)]\times[\hat{\vec{p}}-q\vec{A}(\hat{\vec{r}},t)]\psi
\end{align}
Plugging the equation above into the equation (\ref{operate}) gives
\begin{equation}
\hat{H}\psi=\frac{1}{2m}\{[\hat{\vec{p}}-q\vec{A}(\hat{\vec{r}},t)]^2+i\vec{\sigma}\cdot[\hat{\vec{p}}-q\vec{A}(\hat{\vec{r}},t)]\times[\hat{\vec{p}}-q\vec{A}(\hat{\vec{r}},t)]\}\psi+qU(\hat{\vec{r}},t)\psi
\end{equation}
Using
\begin{equation}
(\vec{\sigma}\cdot\vec{A})(\vec{\sigma}\cdot\vec{B})=\vec{A}\times\vec{B}+i\vec{\sigma}\cdot\vec{A}\times\vec{B}
\end{equation}
we have
\begin{equation}
\hat{H}\psi=\frac{1}{2m}\{\vec{\sigma}\cdot[\vec{\hat{p}}-q\vec{A}(\hat{\vec{r}},t)]\}^2\psi+qU(\hat{\vec{r}},t)
\end{equation}
Therefore, the Hamiltonian can be written in the form
\begin{equation}
\hat{H}=\frac{1}{2m}\{\vec{\sigma}\cdot[\hat{\vec{p}}-q\vec{A}(\hat{\vec{r}},t)]\}^2+qU(\hat{\vec{r}},t)
\end{equation}
\end{sol}

\begin{problem}{4}
[C-T Exercise 10-3] Consider a system composed of two spin $1/2$ particles whose orbital variables are ignored. The Hamiltonian of the system is $\hat{H}=\omega_1\hat{S}_{1z}+\omega_2\hat{S}_{2z}$, where $\hat{S}_{1z}$ and $\hat{S}_{2z}$ are the projections of the spins $\hat{S}_1$ and $\hat{S}_2$ of the two particles onto $Oz$, and $\omega_1$ and $\omega_2$ are real constants.
\begin{itemize}
\item[(a)] The initial state of the system, at time $t=0$, is $|\psi(0)\rangle=\frac{1}{\sqrt{2}}[|+-\rangle+|-+\rangle]$. At time $t$, $\hat{\vec{S}}^2=(\hat{\vec{S}}_1+\hat{\vec{S}}_2)^2$ is measured. What results can be found, and with what probabilities?
\item[(b)] If the initial state of the system is arbitrary, what Bohr frequencies can appear in the evolution of $\langle\hat{\vec{S}}^2\rangle$? Same question for $\hat{S}_x=\hat{S}_{1x}+\hat{S}_{2x}$.
\end{itemize}
\end{problem}
\begin{sol}
\begin{itemize}
\item[(a)] The energy of the two eigenstates $|+-\rangle$ and $|-+\rangle$ are
\begin{gather}
E_{+-}=\langle+-\hat{H}|+-\rangle=\langle+-|(\omega_1\hat{S}_{1z}+\omega_2\hat{S}_{2z})|+-\rangle=\frac{\hbar}{2}(\omega_1-\omega_2)\\
E_{-+}=\langle-+\hat{H}|-+\rangle=\langle-+|(\omega_1\hat{S}_{1z}+\omega_2\hat{S}_{2z})|-+\rangle=\frac{\hbar}{2}(-\omega_1+\omega_2)
\end{gather}
At time $t$, the state of the system is
\begin{align}
\nonumber|\psi(t)\rangle=&\frac{1}{\sqrt{2}}[e^{iE_{+-}t/\hbar}|+-\rangle+e^{iE_{-+}t/\hbar}|-+\rangle]\\
=&\frac{1}{\sqrt{2}}[e^{i(\omega_1-\omega_2)t/2}|+-\rangle+e^{i(-\omega_1+\omega_2)t/2}|-+\rangle]
\end{align}
If $\hat{\vec{S}}^2$ is measured, result $2\hbar^2$ can be found with probability
\begin{align}
\nonumber\mathscr{P}(\hat{\vec{S}}^2=2\hbar^2)=&|\langle11|\psi(t)\rangle|^2+|\langle10|\psi(t)\rangle|^2+|\langle1,-1|\psi(t)\rangle|^2\\
\nonumber=&\frac{1}{2}[|\langle++|(e^{i(\omega_1-\omega_2)t/2}|+-\rangle+e^{i(-\omega_1+\omega_2)t/2}|-+\rangle)|^2\\
\nonumber&+|\frac{1}{\sqrt{2}}(\langle+-|+\langle+-|)(e^{i(\omega_1-\omega_2)t/2}|+-\rangle+e^{i(-\omega_1+\omega_2)t/2}|-+\rangle)|^2\\
\nonumber&+|\langle--|(e^{i(\omega_1-\omega_2)t/2}|+-\rangle+e^{i(-\omega_1+\omega_2)t/2}|-+\rangle)|^2]\\
\nonumber=&\frac{1}{2}|\frac{1}{\sqrt{2}}(e^{i(\omega_1-\omega_2)t/2}+e^{i(-\omega_1+\omega_2)t/2})|^2\\
=&\cos^2\frac{\omega_1-\omega_2}{2}t
\end{align}
Result $0$ can be found with probability
\begin{align}
\nonumber\mathscr{P}(\hat{\vec{S}}^2=0)=1-\mathscr{P}(\hat{\vec{S}}^2=2\hbar^2)=\sin^2\frac{\omega_1-\omega_2}{2}t
\end{align}
\item[(b)] Using Ehrenfest Theorem we have
\begin{equation}
\frac{d\langle\hat{\vec{S}}^2\rangle}{dt}=\frac{1}{i\hbar}\langle[\hat{H},\hat{\vec{S}}^2]\rangle
\end{equation}
Since
\begin{align}
\nonumber[\hat{H},\hat{\vec{S}}^2]=&[\omega_1\hat{S}_{1z}+\omega_2\hat{S}_{2z},(\hat{\vec{S}}_1+\hat{\vec{S}}_2)^2]\\
\nonumber=&\omega_1\{[\hat{S}_{1z},\hat{\vec{S}}_1^2]+[\hat{S}_{1z},\hat{\vec{S}}_1\hat{\vec{S}}_2]+[\hat{S}_{1z},\hat{\vec{S}}_2\hat{\vec{S}}_1]+[\hat{S}_{1z},\hat{\vec{S}}_2^2]\}\\
\nonumber&+\omega_2\{[\hat{S}_{2z},\hat{\vec{S}}_1^2]+[\hat{S}_{2z},\hat{\vec{S}}_1\hat{\vec{S}}_2]+[\hat{S}_{2z},\hat{\vec{S}}_2\hat{\vec{S}}_1]+[\hat{S}_{2z},\hat{\vec{S}}_2^2]\}\\
=&0
\end{align}
$\hat{H}$ and $\hat{\vec{S}}^2$ commute, we have
\begin{equation}
\frac{d\langle\hat{\vec{S}}^2\rangle}{dt}=0
\end{equation}
Therefore, $\langle\hat{\vec{S}}^2\rangle$ remains a constant and does not evolve with time and no Bohr frequency can appear.\\
As for the same question for $\langle\hat{S}_x\rangle=\langle\hat{S}_{1x}+\hat{S}_{2x}\rangle$: the initial arbitrary state of the system can be written as
\begin{equation}
|\psi(0)\rangle=\alpha|++\rangle+\beta|+-\rangle+\gamma|-+\rangle+\delta|--\rangle
\end{equation}
The energy of the four eigenstates are
\begin{gather}
E_{++}=\frac{\hbar}{2}(\omega_1+\omega_2)\\
E_{+-}=\frac{\hbar}{2}(\omega_1-\omega_2)\\
E_{-+}=\frac{\hbar}{2}(-\omega_1+\omega_2)\\
E_{--}=\frac{\hbar}{2}(-\omega_1-\omega_2)
\end{gather}
The state of the system at time $t$ is
\begin{align}
\nonumber|\psi(t)\rangle=&\alpha e^{iE_{++}t/\hbar}|++\rangle+\beta e^{iE_{+-}t/\hbar}|+-\rangle+\gamma e^{iE_{-+}t/\hbar}|-+\rangle+\delta e^{iE_{--}t/\hbar}|--\rangle\\
=&\alpha e^{i(\omega_1+\omega_2)t/2}|++\rangle+\beta e^{i(\omega_1-\omega_2)t/2}|+-\rangle+\gamma e^{i(-\omega_1+\omega_2)t/2}|-+\rangle+\delta e^{i(-\omega_1-\omega_2)t/2}|--\rangle
\end{align}
\begin{align}
\nonumber\langle\hat{S}_x\rangle=&\langle\psi(t)|\hat{S}_x|\psi(t)\rangle\\
\nonumber=&(\alpha^*e^{-i(\omega_1+\omega_2)t/2}\langle++|+\beta^*e^{-i(\omega_1-\omega_2)t/2}\langle+-|+\gamma^*e^{-i(-\omega_1+\omega_2)t/2}\langle-+|+\delta^*e^{-i(-\omega_1-\omega_2)t/2}\langle--|)\\
\nonumber&(\hat{S}_{1x}+\hat{S}_{2x})\\
\nonumber&(\alpha e^{i(\omega_1+\omega_2)t/2}|++\rangle+\beta e^{i(\omega_1-\omega_2)t/2}|+-\rangle+\gamma e^{i(-\omega_1+\omega_2)t/2}|-+\rangle+\delta e^{i(-\omega_1-\omega_2)t/2}|--\rangle)\\
\nonumber=&(\alpha^*e^{-i(\omega_1+\omega_2)t/2}\langle++|+\beta^*e^{-i(\omega_1-\omega_2)t/2}\langle+-|+\gamma^*e^{-i(-\omega_1+\omega_2)t/2}\langle-+|+\delta^*e^{-i(-\omega_1-\omega_2)t/2}\langle--|)\\
\nonumber&(\frac{\hat{S}_{1+}+\hat{S}_{1-}}{2}+\frac{\hat{S}_{2+}+\hat{S}_{2-}}{2})\\
\nonumber&(\alpha e^{i(\omega_1+\omega_2)t/2}|++\rangle+\beta e^{i(\omega_1-\omega_2)t/2}|+-\rangle+\gamma e^{i(-\omega_1+\omega_2)t/2}|-+\rangle+\delta e^{i(-\omega_1-\omega_2)t/2}|--\rangle)\\
\nonumber=&\frac{\hbar}{2}[(\alpha^*\gamma e^{-i\omega_1t}+\alpha\gamma^*e^{i\omega_1t})+(\beta^*\delta e^{-i\omega_1t}+\beta\delta^*e^{i\omega_1t})+(\alpha^*\beta e^{-i\omega_2t}+\alpha\beta^*e^{i\omega_2t})+(\gamma^*\delta e^{-i\omega_2t}+\gamma\delta^*e^{i\omega_2t})]\\
\nonumber=&\frac{\hbar}{2}[(\text{Re}(\alpha^*\gamma)+\text{Re}(\beta^*\delta))\cos\omega_1t+(\text{Im}(\alpha^*\gamma)+\text{Im}(\beta^*\delta))\sin\omega_1t\\
&+(\text{Re}(\alpha^*\beta)+\text{Re}(\gamma^*\delta))\cos\omega_2t+(\text{Im}(\alpha^*\beta)+\text{Im}(\gamma^*\delta))\sin\omega_2t]
\end{align}
Both $\omega_1$ and $\omega_2$ can appear in the revolution of $\langle\hat{S}_x\rangle$.
\end{itemize}
\small{Reference: https://phys.cst.temple.edu/$\sim$meziani/homework3s\_5702\_2016.pdf}
\end{sol}

\begin{problem}{5}
[C-T Exercise 10-5] Let $\hat{\vec{S}}=\hat{\vec{S}}_1+\hat{\vec{S}}_2+\hat{\vec{S}}_3$ be the total angular momentum of three spin $1/2$ particles (whose orbital variables will be ignored). Let $|\varepsilon_1\varepsilon_2\varepsilon_3\rangle$ be the eigenstates common to $\hat{S}_{1z}$, $\hat{S}_{2z}$, and $\hat{S}_{3z}$, of respective eigenvalues $\varepsilon_1\hbar/2$, $\varepsilon_2\hbar/2$, and $\varepsilon_3\hbar/2$. Give a basis of eigenvectors common to $\hat{\vec{S}}^2$ and $\hat{S}_z$, in terms of the kets $|\varepsilon_1\varepsilon_2\varepsilon_3\rangle$. Do these two operators form a CSCO? (Begin by adding two of the spins, then add the partial angular momentum so obtained to the third one.)
\end{problem}
\begin{sol}
Add two of the spin $\hat{\vec{S}}_1$ and $\hat{\vec{S}}_2$ first. Let $\hat{S}_{12}=\hat{\vec{S}}_{1}+\hat{\vec{S}}_2$, the eigenvecors common to $\hat{\vec{S}}_{12}^2$ and $\hat{\vec{S}}_{12z}$ are $|s_{12}m_{12}\rangle$.
\begin{gather}
\hat{\vec{S}}^2|s_{12}m_{12}\rangle=s_{12}(s_{12}+1)\hbar^2|s_{12}m_{12}\rangle\\
\hat{S}_{12z}|s_{12}m_{12}\rangle=m_{12}\hbar|s_{12}m_{12}\rangle
\end{gather}
In subspace $\mathscr{E}(s_{12}=1)$, $m_{12}=1,0,-1$. Let
\begin{equation}
|11\rangle=|++\rangle
\end{equation}
Then
\begin{equation}
\hat{S}_{12-}|11\rangle=\hbar\sqrt{1(1+1)-1(1-1)}|10\rangle=\hbar\sqrt{2}|10\rangle
\end{equation}
so
\begin{equation}
|10\rangle=\frac{1}{\hbar\sqrt{2}}\hat{S}_{12-}|11\rangle=\frac{1}{\hbar\sqrt{2}}(\hat{S}_{1-}+\hat{S}_{2-})|++\rangle=\frac{1}{\sqrt{2}}[|+-\rangle+|-+\rangle]
\end{equation}
And
\begin{equation}
\hat{S}_{12-}|10\rangle=\hbar\sqrt{1(1+1)-0(0-1)}|1,-1\rangle=\hbar\sqrt{2}|1,-1\rangle
\end{equation}
so
\begin{equation}
|1,-1\rangle=\frac{1}{\hbar\sqrt{2}}\hat{S}_{12-}|10\rangle=\frac{1}{2\hbar}(\hat{S}_{1-}+\hat{S}_{2-})[|+-\rangle+|-+\rangle]=|--\rangle
\end{equation}
In subspace $\mathscr{E}(s_{12}=0)$, $m_{12}=0$. Since $|00\rangle$ is orthogonal to $|11\rangle=|++\rangle$ and $|1,-1\rangle=|--\rangle$, it is in the form
\begin{equation}
|00\rangle=a_1|+-\rangle+a_2|-+\rangle
\end{equation}
Since $|00\rangle$ is orthogonal to $|10\rangle=\frac{1}{\sqrt{2}}[|+-\rangle+|-+\rangle]$ and the normalization condition requires $|\alpha|^2+|\beta|^2=1$,
\begin{equation}
|00\rangle=\frac{1}{\sqrt{2}}[|+-\rangle-|-+\rangle]
\end{equation}
Now add the partial angular momentum to obtain the third one:
Wirte the eigenvectors common to $\hat{\vec{S}}^2$ and $\hat{S}_z$ as $|sm\rangle$.
\begin{equation}
\hat{\vec{S}}^2|sm\rangle=s(s+1)\hbar^2|sm\rangle\\
\hat{S}_z|sm\rangle=m\hbar|sm\rangle
\end{equation}
In subspace $\mathscr{E}(s=\frac{3}{2})$, $m=\frac{3}{2},\frac{1}{2},-\frac{1}{2},-\frac{3}{2}$. Let
\begin{equation}
|\frac{3}{2}\frac{3}{2}\rangle=|+++\rangle
\end{equation}
Then
\begin{equation}
\hat{S}_-|\frac{3}{2}\frac{3}{2}\rangle=\hbar\sqrt{\frac{3}{2}(\frac{3}{2}+1)-\frac{3}{2}(\frac{3}{2}-1)}|\frac{3}{2}\frac{1}{2}\rangle=\hbar\sqrt{3}|\frac{3}{2}\frac{1}{2}\rangle
\end{equation}
so
\begin{equation}
|\frac{3}{2}\frac{1}{2}\rangle=\frac{1}{\hbar\sqrt{2}}\hat{S}_-|11\rangle=\frac{1}{\hbar\sqrt{3}}(\hat{S}_{1-}+\hat{S}_{2-}+\hat{S}_{3-})|+++\rangle=\frac{1}{\sqrt{3}}[|++-\rangle+|+-+\rangle+|-++\rangle]
\end{equation}
And
\begin{equation}
\hat{S}_-|\frac{3}{2},\frac{1}{2}\rangle=\hbar\sqrt{\frac{3}{2}(\frac{3}{2}+1)-\frac{1}{2}(\frac{1}{2}-1)}|\frac{3}{2},-\frac{1}{2}\rangle=2\hbar|\frac{3}{2},-\frac{1}{2}\rangle
\end{equation}
so
\begin{align}
\nonumber|\frac{3}{2},-\frac{1}{2}\rangle=&\frac{1}{2\hbar}\hat{S}_-|\frac{3}{2}\frac{1}{2}\rangle=\frac{1}{2\sqrt{3}\hbar}(\hat{S}_{1-}+\hat{S}_{2-}+\hat{S}_{3-})[|++-\rangle+|+-+\rangle+|-++\rangle]\\
=&\frac{1}{\sqrt{3}}[|+--\rangle+|-+-\rangle+|--+\rangle]
\end{align}
And
\begin{equation}
\hat{S}_-|\frac{3}{2},-\frac{1}{2}\rangle=\hbar\sqrt{\frac{3}{2}(\frac{3}{2}+1)-(-\frac{1}{2})(-\frac{1}{2}-1)}=\hbar\sqrt{3}|\frac{3}{2},-\frac{3}{2}\rangle
\end{equation}
so
\begin{equation}
|\frac{3}{2},-\frac{3}{2}\rangle=\frac{1}{\hbar\sqrt{3}}\hat{S}_-|\frac{3}{2},-\frac{1}{2}\rangle=\frac{1}{3\hbar}(\hat{S}_{1-}+\hat{S}_{2-}+\hat{S}_{3-})[|+--\rangle+|-+-\rangle+|--+\rangle]=|---\rangle
\end{equation}
In subspace $\mathscr{E}(s=\frac{1}{2})$, $m=\frac{1}{2},-\frac{1}{2}$. $|\frac{1}{2}\frac{1}{2}\rangle$ is orthogonal to $|\frac{3}{2}\frac{3}{2}\rangle$, $|\frac{3}{2}\frac{1}{2}\rangle$, $|\frac{3}{2},-\frac{1}{2}\rangle$, $|\frac{3}{2},-\frac{3}{2}\rangle$ and is normalized, so it can be
\begin{equation}
|\frac{1}{2}\frac{1}{2}\rangle_1=\frac{1}{\sqrt{2}}[|++-\rangle-|-++\rangle]
\end{equation}
or
\begin{equation}
|\frac{1}{2}\frac{1}{2}\rangle_2=-\frac{1}{\sqrt{6}}|++-\rangle+\frac{2}{\sqrt{6}}|+-+\rangle-\frac{1}{\sqrt{6}}|-++\rangle
\end{equation}
(degree of degeneracy is $2$.)\\
Then
\begin{equation}
\hat{S}_-|\frac{1}{2}\frac{1}{2}\rangle_{1,2}=\sqrt{\frac{1}{2}(\frac{1}{2}+1)-\frac{1}{2}(\frac{1}{2}-\frac{1}{2})}=\hbar|\frac{1}{2},-\frac{1}{2}\rangle_{1,2}
\end{equation}
so
\begin{equation}
|\frac{1}{2},-\frac{1}{2}\rangle_1=\frac{1}{\hbar\sqrt{2}}\hat{S}_-|\frac{1}{2}\frac{1}{2}\rangle_1=\frac{1}{\hbar\sqrt{2}}(\hat{S}_{1-}+\hat{S}_{2-}+\hat{S}_{3-})[|++-\rangle-|-++\rangle]=\frac{1}{\sqrt{2}}[|+--\rangle-|--+\rangle]
\end{equation}
or
\begin{align}
\nonumber|\frac{1}{2},-\frac{1}{2}\rangle_2=&\frac{1}{\hbar}\hat{S}_-|\frac{1}{2}\frac{1}{2}\rangle=\frac{1}{\hbar}(\hat{S}_{1-}+\hat{S}_{2-}+\hat{S}_{3-})[-\frac{1}{\sqrt{6}}|++-\rangle+\frac{2}{\sqrt{6}}|+-+\rangle-\frac{1}{\sqrt{6}}|-++\rangle]\\
=&\frac{1}{\sqrt{6}}|+--\rangle-\frac{2}{\sqrt{6}}|-+-\rangle+\frac{1}{\sqrt{6}}|--+\rangle
\end{align}
These two operators do not form a CSCO.
\end{sol}
\end{document}