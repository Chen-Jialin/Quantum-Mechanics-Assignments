% !TEX program = pdflatex
% Quantum Mechanics Homework_6
\documentclass[12pt,a4paper]{article}
\usepackage[margin=1in]{geometry} 
\usepackage{amsmath,amsthm,amssymb,amsfonts,enumitem,fancyhdr,color,comment,graphicx,environ}
\pagestyle{fancy}
\setlength{\headheight}{65pt}
\newenvironment{problem}[2][Problem]{\begin{trivlist}
\item[\hskip \labelsep {\bfseries #1}\hskip \labelsep {\bfseries #2.}]}{\end{trivlist}}
\newenvironment{sol}
    {\emph{Solution:}
    }
    {
    \qed
    }
\specialcomment{com}{ \color{blue} \textbf{Comment:} }{\color{black}} %for instructor comments while grading
\NewEnviron{probscore}{\marginpar{ \color{blue} \tiny Problem Score: \BODY \color{black} }}
\usepackage[UTF8]{ctex}
\lhead{Name: 陈稼霖\\ StudentID: 45875852}
\rhead{PHYS1501 \\ Quantum Mechanics \\ Semester Fall 2019 \\ Assignment 6}
\begin{document}
\begin{problem}{1}
In a given representation, the matrix representing the Hamiltonian of a particle is given by
\[
H=\hbar\omega_0\left(\begin{array}{cccccc}
-1+\varepsilon&0&0&0&0&0\\
0&-1-\varepsilon&\sqrt{2}\varepsilon&0&0&0\\
0&\sqrt{2}\varepsilon&-1&0&0&0\\
0&0&0&-1&\sqrt{2}\varepsilon&0\\
0&0&0&\sqrt{2}\varepsilon&-1-\varepsilon&0\\
0&0&0&0&0&-1+\varepsilon\\
\end{array}\right)
\]
with $0<\varepsilon<1$. Find the energy eigenvalues and eigenfunctions of the particle in the representation.
\end{problem}
\begin{sol}
For convenience, let eigenvalue $\lambda=\hbar\omega_0\lambda'$.The characteristic equation
\begin{align}
\nonumber|H-\lambda I|=&\hbar\omega_0\left|\begin{array}{cccccc}
-1+\varepsilon-\lambda'&0&0&0&0&0\\
0&-1-\varepsilon-\lambda'&\sqrt{2}\varepsilon&0&0&0\\
0&\sqrt{2}\varepsilon&-1-\lambda'&0&0&0\\
0&0&0&-1-\lambda'&\sqrt{2}\varepsilon&0\\
0&0&0&\sqrt{2}\varepsilon&-1-\varepsilon-\lambda'&0\\
0&0&0&0&0&-1+\varepsilon-\lambda'\\
\end{array}\right|\\
\nonumber=&\hbar\omega_0[(-1+\varepsilon-\lambda')(-1-\varepsilon-\lambda')(-1-\lambda')(-1-\lambda')(-1-\varepsilon-\lambda')(-1+\varepsilon-\lambda')\\
\nonumber&-(-1+\varepsilon-\lambda')\sqrt{2}\varepsilon\sqrt{2}\varepsilon(-1-\lambda')(-1-\varepsilon-\lambda')(-1+\varepsilon-\lambda')\\
\nonumber&-(-1+\varepsilon-\lambda')(-1-\varepsilon-\lambda')(-1-\lambda')\sqrt{2}\varepsilon\sqrt{2}\varepsilon(-1+\varepsilon-\lambda')\\
\nonumber&+(-1+\varepsilon-\lambda')\sqrt{2}\varepsilon\sqrt{2}\varepsilon\sqrt{2}\varepsilon\sqrt{2}\varepsilon(-1+\varepsilon-\lambda')]\\
=&\hbar\omega_0(-1+\varepsilon-\lambda')^4(-1-2\varepsilon-\lambda')^2=0
\end{align}
gives
\begin{equation}
\lambda_1'=\lambda_2'=\lambda_3'=\lambda_4'=-1+\varepsilon,\quad\lambda_5'=\lambda_6'=-1-2\varepsilon
\end{equation}
so the eigenvalues are
\begin{equation}
\lambda_1=\lambda_2=\lambda_3=\lambda_4=\hbar\omega_0(-1+\varepsilon),\quad\lambda_3=\lambda_4=\hbar\omega_0(-1-2\varepsilon)
\end{equation}
When the eigenvalue $\lambda=\hbar\omega_0(-1+\varepsilon)$,
\begin{equation}
(H-\lambda I)\psi=\hbar\omega_0\left(\begin{array}{cccccc}
0&0&0&0&0&0\\
0&-2\varepsilon&\sqrt{2}\varepsilon&0&0&0\\
0&\sqrt{2}\varepsilon&-\varepsilon&0&0&0\\
0&0&0&-\varepsilon&\sqrt{2}\varepsilon&0\\
0&0&0\sqrt{2}\varepsilon&-2\varepsilon&0\\
0&0&0&0&0&0\\
\end{array}\right)\left(\begin{array}{c}
c_1\\
c_2\\
c_3\\
c_4\\
c_5\\
c_6\\
\end{array}\right)=\left(\begin{array}{c}
0\\
0\\
0\\
0\\
0\\
0\\
\end{array}\right)
\end{equation}
gives four independent normalized eigenvectors
\begin{equation}
\psi_1=\frac{1}{\sqrt{7}}\left(\begin{array}{c}
1\\
1\\
\sqrt{2}\\
\sqrt{2}\\
1\\
0\\
\end{array}\right),\quad
\psi_2=\frac{1}{\sqrt{7}}\left(\begin{array}{c}
0\\
1\\
\sqrt{2}\\
\sqrt{2}\\
1\\
1\\
\end{array}\right),\quad
\psi_3=\frac{1}{\sqrt{7}}\left(\begin{array}{c}
1\\
1\\
\sqrt{2}\\
-\sqrt{2}\\
-1\\
0\\
\end{array}\right),\quad
\psi_4=\frac{1}{\sqrt{7}}\left(\begin{array}{c}
0\\
1\\
\sqrt{2}\\
-\sqrt{2}\\
-1\\
1\\
\end{array}\right)
\end{equation}
When eigenvalue $\lambda=\hbar\omega_0(-1-2\varepsilon)$,
\begin{equation}
(H-\lambda I)\psi=\hbar\omega_0\left(\begin{array}{cccccc}
3\varepsilon&0&0&0&0&0\\
0&\varepsilon&\sqrt{2}\varepsilon&0&0&0\\
0&\sqrt{2}\varepsilon&2\varepsilon&0&0&0\\
0&0&0&2\varepsilon&\sqrt{2}\varepsilon&0\\
0&0&0\sqrt{2}\varepsilon&2\varepsilon&0\\
0&0&0&0&0&3\varepsilon\\
\end{array}\right)\left(\begin{array}{c}
c_1\\
c_2\\
c_3\\
c_4\\
c_5\\
c_6\\
\end{array}\right)=\left(\begin{array}{c}
0\\
0\\
0\\
0\\
0\\
0\\
\end{array}\right)
\end{equation}
gives two independent normalized eigenvectors
\begin{equation}
\psi_5=\frac{1}{\sqrt{6}}\left(\begin{array}{c}
0\\
\sqrt{2}\\
-1\\
-1\\
\sqrt{2}\\
0\\
\end{array}\right),\quad
\psi_6=\frac{1}{\sqrt{6}}\left(\begin{array}{c}
0\\
\sqrt{2}\\
-1\\
1\\
-\sqrt{2}\\
0\\
\end{array}\right)
\end{equation}
\end{sol}

\begin{problem}{2}
[C-T exercise 2-4] Let $\hat{K}$ be the operator defined by $\hat{K}=|\varphi\rangle\langle\psi|$, where $|\varphi\rangle$ and $|\psi\rangle$ are two vectors of the state space.
\begin{itemize}
\item[(a)] Under what condition is $\hat{K}$ Hermitian?
\item[(b)] Calculate $\hat{K}^2$. Under what condition is $\hat{K}$ a projector?
\item[(c)] Show that $\hat{K}$ can always be written in the form $\hat{K}=\lambda\hat{P}_1\hat{P}_2$ where $\lambda$ is a constant to be calculated and $\hat{P}_1$ and $\hat{P}_2$ are projectors.
\end{itemize}
\end{problem}
\begin{sol}
\begin{itemize}
\item[(a)] The definition of Hermitian
\begin{equation}
\hat{K}=\hat{K}^{\dagger}
\end{equation}
Plugging in definition of $\hat{H}$, we get
\begin{equation}
|\varphi\rangle\langle\psi|=(|\varphi\rangle\langle\psi|)^{\dagger}
\end{equation}
Using the relation
\begin{equation}
(|\varphi\rangle\langle\psi|)^{\dagger}=|\psi\rangle\langle\varphi|
\end{equation}
we get
\begin{equation}
|\varphi\rangle\langle\psi|=|\psi\rangle\langle\varphi|
\end{equation}
Therefore, $\hat{H}$ is Hermitian when $|\varphi\rangle\langle\psi|=|\psi\rangle\langle\varphi|$.
\item[(b)]
\begin{equation}
\hat{K}^2=|\varphi\rangle\langle\psi|\varphi\rangle\langle\psi|
\end{equation}
The definition of projector
\begin{gather}
\hat{K}^2=|\varphi\rangle\langle\psi|\varphi\rangle\langle\psi|=\hat{K}=|\varphi\rangle\langle\psi|\\
\Longrightarrow\langle\psi|\varphi\rangle=1
\end{gather}
Therefore, $\hat{K}$ is a projector when $\langle\psi|\varphi\rangle=1$.
\item[(c)] Rewrite $\hat{K}$ as
\begin{equation}
\hat{K}=\frac{|\varphi\rangle\langle\varphi|\psi\rangle\langle\psi|}{\langle\varphi|\psi\rangle}
\end{equation}
Therefore, $\hat{K}$ can always be written in the form
\begin{equation}
\hat{K}=\lambda\hat{P}_1\hat{P}_2
\end{equation}
where
\begin{equation}
\lambda=\frac{1}{\langle\varphi|\psi\rangle}
\end{equation}
is a constant to be calculated and
\begin{gather}
\hat{P}_1=|\varphi\rangle\langle\varphi|\\
\hat{P}_2=|\psi\rangle\langle\psi|
\end{gather}
are projectors.
\end{itemize}
\end{sol}

\begin{problem}{3}
[C-T exercise 2-5] Let $\hat{P}_1$ be the orthogonal projector onto the subspace $\mathcal{E}_1$, $\hat{P}_2$ the orthogonal projector on to the subspace $\mathcal{E}_2$. Show that, for the product $\hat{P}_1\hat{P}_2$ to be an orthogonal projector as well, it is necessary and sufficient that $\hat{P}_1$ and $\hat{P}_2$ commute. In this case, what is the subspace onto with $\hat{P}_1\hat{P}_2$ projects?
\end{problem}
\begin{sol}
% Suppose the subspace $\mathcal{E}_1$ is spanned by orthonormal vectors $|\varphi_1^{1}\rangle,|\varphi_1^{2}\rangle,\cdots,|\varphi_1^{g_n}\rangle$ while the subspace $\mathcal{E}_2$ is spanned by othornormal vectors $|\varphi_2^{1}\rangle,|\varphi_2^{2}\rangle,\cdots,|\varphi_2^{g_m}\rangle$. Then the orthogonal projectors are represented as
% \begin{gather}
% \hat{P}_1=\sum_{i=1}^{g_n}|\varphi_1^i\rangle\langle\varphi_1^i|\\
% \hat{P}_2=\sum_{j=1}^{g_m}|\varphi_2^j\rangle\langle\varphi_2^j|
% \end{gather}
% The product
% \begin{equation}
% \hat{P}_1\hat{P}_2=\sum_{i=1}^{g_n}|\varphi_1^i\rangle\langle\varphi_1^i|\sum_{j=1}^{g_m}|\varphi_2^j\rangle\langle\varphi_2^j|
% \end{equation}
Since $\hat{P}_1$ and $\hat{P}_2$ are orthogonal projectors,
\begin{align}
\text{orthogonal: }&\hat{P}_1\hat{P}_1^T=I\\
&\hat{P}_2\hat{P}_2^T=I\\
\text{projector: }&\hat{P}_1\hat{P}_1=\hat{P}\\
&\hat{P}_2\hat{P}_2=\hat{P}_2
\end{align}

First, let's show that $\hat{P}_1$ and $\hat{P}_2$ commute is necessary for the product $\hat{P}_1\hat{P}_2$ to be orthogonal projector:

Suppose $\hat{P}_1\hat{P}_2$ is orthogonal projector,
\begin{align}
\text{orthogonal: }&(\hat{P}_1\hat{P}_2)(\hat{P}_1\hat{P}_2)^T=I\\
\label{projector}\text{projector: }&(\hat{P}_1\hat{P}_2)(\hat{P}_1\hat{P}_2)=(\hat{P}_1\hat{P}_2)
\end{align}
Premultiplying the equation (\ref{projector}) with $\hat{P}_1^T\hat{P}_2^T$ at both side gives
\begin{equation}
\hat{P}_1\hat{P}_2=\hat{P}_2^T\hat{P}_1^T(\hat{P}_1\hat{P}_2)(\hat{P}_1\hat{P}_2)=\hat{P}_2^T\hat{P}_1^T(\hat{P}_1\hat{P}_2)=I
\end{equation}
Premultiplying the equation (\ref{projector}) with $\hat{P}_1^T$ and postmultiplying it with $\hat{P}_2^T$ at both side gives
\begin{equation}
\hat{P}_2\hat{P}_1=\hat{P}_1^T(\hat{P}_1\hat{P}_2)(\hat{P}_1\hat{P}_2)\hat{P}_2^T=\hat{P}_1^T(\hat{P}_1\hat{P}_2)\hat{P}_2^T=I
\end{equation}
So
\begin{equation}
[\hat{P}_1,\hat{P}_2]=\hat{P}_1\hat{P}_2-\hat{P}_2\hat{P}_1=I-I=0
\end{equation}
$\hat{P}_1$ and $\hat{P}_2$ commute.

Next, let's show that $\hat{P}_1$ and $\hat{P}_2$ commute is sufficient for the product $\hat{P}_1\hat{P}_2$ to be orthogonal projector:

Suppose $\hat{P}_1$ and $\hat{P}_2$ commute,
\begin{gather}
[\hat{P}_1,\hat{P}_2]=\hat{P}_1\hat{P}_2-\hat{P}_2\hat{P}_1=0\\
\Longrightarrow\hat{P}_1\hat{P}_2=\hat{P}_2\hat{P}_1
\end{gather}
Then we have
\begin{equation}
(\hat{P}_1\hat{P}_2)(\hat{P}_1\hat{P}_2)=\hat{P}_1\hat{P}_2\hat{P}_1\hat{P}_2=\hat{P}_1\hat{P}_1\hat{P}_2\hat{P}_2=(\hat{P}_1\hat{P}_1)(\hat{P}_2\hat{P}_2)=\hat{P}_1\hat{P}_2
\end{equation}
so $\hat{P}_1\hat{P}_2$ is a projector.\\
And
\begin{equation}
(\hat{P}_1\hat{P}_2)(\hat{P}_1\hat{P}_2)^T=\hat{P}_1\hat{P}_2\hat{P}_2^T\hat{P}_1^T=\hat{P}_1(\hat{P}_2\hat{P}_2^T)\hat{P}_1^T=\hat{P}_1\hat{P}_1^T=I
\end{equation}
so $\hat{P}_1\hat{P}_2$ is orthogonal.

Therefore, for the product $\hat{P}_1\hat{P}_2$ to be an orthogonal projector as well, it is necessary and sufficient that $\hat{P}_1$ and $\hat{P}_2$ commute.

The subspace onto which $\hat{P}_1\hat{P}_2$ projects is $\mathcal{E}_1\otimes\mathcal{E}_2$
\end{sol}

\begin{problem}{4}
[C-T exercise 2-11] Consider a physical system whose three-dimensional state space is spanned by the orthogonal basis formed by the three kets $|u_1\rangle$, $|u_2\rangle$, and $|u_3\rangle$.  In the basis of these three vectors, taken in this order,  the two operators $\hat{H}$ and $\hat{B}$ are defined by
\[
H=\hbar\omega_0\left(\begin{array}{ccc}
1&0&0\\
0&-1&0\\
0&0&-1\\
\end{array}\right),\quad B=b\left(\begin{array}{ccc}
1&0&0\\
0&0&1\\
0&1&0\\
\end{array}\right)
\]
where $\omega_0$ and $b$ are real constants.
\begin{itemize}
\item[(a)] Are $H$ and $B$ Hermitian?
\item[(b)] Show that $H$ and $B$ commute. Give a basis of eigenvectors common to $H$ and $B$.
\end{itemize}
\end{problem}
\begin{sol}
\begin{itemize}
\item[(a)]
\begin{equation}
H^{\dagger}=(H^T)^*=\hbar\omega\left(\begin{array}{ccc}
1&0&0\\
0&-1&0\\
0&0&-1\\
\end{array}\right)=H
\end{equation}
so $H$ is Hermitian.
\begin{equation}
B^{\dagger}=(B^T)^*=b\left(\begin{array}{ccc}
1&0&0\\
0&0&1\\
0&1&0\\
\end{array}\right)=B
\end{equation}
so $B$ is Hermitian.
\item[(b)]
\begin{align}
\nonumber[H,B]=&HB-BH\\
\nonumber=&\hbar\omega_0b\left(\begin{array}{ccc}
1&0&0\\
0&-1&0\\
0&0&-1\\
\end{array}\right)\left(\begin{array}{ccc}
1&0&0\\
0&0&1\\
0&1&0\\
\end{array}\right)-\hbar\omega_0\left(\begin{array}{ccc}
1&0&0\\
0&0&1\\
0&1&0\\
\end{array}\right)\left(\begin{array}{ccc}
1&0&0\\
0&-1&0\\
0&0&-1\\
\end{array}\right)\\
=&\hbar\omega_0\left(\begin{array}{ccc}
1&0&0\\
0&0&-1\\
0&-1&0\\
\end{array}\right)-\hbar\omega_0\left(\begin{array}{ccc}
1&0&0\\
0&0&-1\\
0&-1&0\\
\end{array}\right)=0
\end{align}
so $H$ and $B$ commute.\\
Since $H$ and $B$ commute, they have common eigenspace. Let's first find the eigenvectors of $H$. Let $\lambda_H=\hbar\omega_0\lambda_H'$, then the characteristic equation of $H$
\begin{align}
\nonumber|H-\lambda_HI|=&|H-\hbar\omega\lambda_H'I|=\hbar\omega_0\left|\begin{array}{ccc}
1-\lambda'&0&0\\
0&-1-\lambda'&0\\
0&0&-1-\lambda'\\
\end{array}\right|\\
=&\hbar\omega_0(1-\lambda_H')(-1-\lambda_H')^2=0
\end{align}
gives
\begin{equation}
\lambda_{H1}'=1,\quad\lambda_{H2}'=\lambda_{H3}'=-1
\end{equation}
so the eigenvalues of $H$
\begin{equation}
\lambda_{H1}=\hbar\omega_0,\quad\lambda_{H2}=\lambda_{H3}=-\hbar\omega_0
\end{equation}
When the eigenvalue of $H$ $\lambda_H=\hbar\omega_0$,
\begin{equation}
(H-\lambda_HI)\psi=\hbar\omega_0\left(\begin{array}{ccc}
0&0&0\\
0&-2&0\\
0&0&-2\\
\end{array}\right)\left(\begin{array}{c}
c_1\\
c_2\\
c_3\\
\end{array}\right)=\left(\begin{array}{c}
0\\
0\\
0\\
\end{array}\right)
\end{equation}
gives one normalized eigenvector
\begin{equation}
\psi_{H1}=\left(\begin{array}{c}
1\\
0\\
0\\
\end{array}\right)
\end{equation}
When the eigenvalue of $H$ $\lambda_H=-\hbar\omega_0$,
\begin{equation}
(H-\lambda_HI)\psi=\hbar\omega_0\left(\begin{array}{ccc}
2&0&0\\
0&0&0\\
0&0&0\\
\end{array}\right)\left(\begin{array}{c}
c_1\\
c_2\\
c_3\\
\end{array}\right)=\left(\begin{array}{c}
0\\
0\\
0\\
\end{array}\right)
\end{equation}
gives two independent normalized eigenvectors
\begin{equation}
\psi_{H2}=\left(\begin{array}{c}
0\\
1\\
0\\
\end{array}\right),\quad\psi_{H3}=\left(\begin{array}{c}
0\\
0\\
1\\
\end{array}\right)
\end{equation}
Obviously, $\psi_1$ are also an eigenvector of $B$
\begin{equation}
B\psi_{H1}=b\left(\begin{array}{ccc}
1&0&0\\
0&0&1\\
0&1&0\\
\end{array}\right)\left(\begin{array}{c}
1\\
0\\
0\\
\end{array}\right)=b\left(\begin{array}{c}
1\\
0\\
0\\
\end{array}\right)=b\psi_{H1}
\end{equation}
Let $\lambda_B=b\lambda_B'$, then the characteristic equation of $B$
\begin{align}
\nonumber|B-\lambda_BI|=&|B-b\lambda_B'I|=b\left|\begin{array}{ccc}
1-\lambda_B'&0&0\\
0&-\lambda_B'&1\\
0&1&-\lambda_B'\\
\end{array}\right|\\
=&-b(\lambda_B'-1)^2(\lambda_B'+1)=0
\end{align}
gives
\begin{equation}
\lambda_{B1}'=\lambda_{B2}'=1,\quad\lambda_{B3}'=-1
\end{equation}
so the eigenvalues of $B$
\begin{equation}
\lambda_{B1}'=\lambda_{B2}'=b,\quad\lambda_{B3}'=-b
\end{equation}
When eigenvalue of $B$ $\lambda_B=b$,
\begin{equation}
(B-\lambda_BI)\psi_B=b\left(\begin{array}{ccc}
0&0&0\\
0&-1&1\\
0&1&-1\\
\end{array}\right)\left(\begin{array}{c}
d_1\\
d_2\\
d_3\\
\end{array}\right)=\left(\begin{array}{c}
0\\
0\\
0\\
\end{array}\right)
\end{equation}
Besides, $\psi_{B1}=(1,0,0)$, another normalized eigenvector is
\begin{equation}
\psi_{B2}=\left(\begin{array}{c}
0\\
\frac{1}{\sqrt{2}}\\
\frac{1}{\sqrt{2}}\\
\end{array}\right)
\end{equation}
which can be written as a linear combination of eigenvectors of $H$: $\psi_{B2}=\frac{1}{\sqrt{2}}\psi_{H2}+\frac{1}{\sqrt{2}}\psi_{H3}$.\\
When eigenvalue of $B$ $\lambda_B=-b$
\begin{equation}
(B-\lambda_BI)\psi_B=b\left(\begin{array}{ccc}
2&0&0\\
0&1&1\\
0&1&1\\
\end{array}\right)\left(\begin{array}{c}
d_1\\
d_2\\
d_3\\
\end{array}\right)=\left(\begin{array}{c}
0\\
0\\
0\\
\end{array}\right)
\end{equation}
gives one normalized eigenvector
\begin{equation}
\psi_{B2}=\left(\begin{array}{c}
0\\
\frac{1}{\sqrt{2}}\\
-\frac{1}{\sqrt{2}}\\
\end{array}\right)
\end{equation}
which can be written as a linear combination of eigenvectors of $H$: $\psi_{B3}=\frac{1}{\sqrt{2}}\psi_{H2}-\frac{1}{\sqrt{2}}\psi_{H3}$

Therefore, $\left\{\left(\begin{array}{c}1\\0\\0\end{array}\right),\left(\begin{array}{c}0\\\frac{1}{\sqrt{2}}\\\frac{1}{\sqrt{2}}\end{array}\right),\left(\begin{array}{c}0\\\frac{1}{\sqrt{2}}\\-\frac{1}{\sqrt{2}}\end{array}\right)\right\}$ is a basis of eigenvectors common to $H$ and $B$.
\end{itemize}
\end{sol}

\begin{problem}{5}
[C-T exercise 2-12] In the same state space as that of the preceding exercise, consider two operators $\hat{L}_z$ and $\hat{S}$ defined by
\begin{gather*}
\hat{L}_z|u_1\rangle=|u_1\rangle,\quad\hat{L}_z|u_2\rangle=0,\quad\hat{L}_z=-|u_3\rangle;\\
\hat{S}|u_1\rangle=|u_3\rangle,\quad\hat{S}|u_2\rangle=|u_2\rangle,\quad\hat{S}|u_3\rangle=|u_1\rangle.
\end{gather*}
\begin{itemize}
\item[(a)] Write the matrices which represent, in the $\{|u_1\rangle,|u_2\rangle,|u_3\rangle\}$ basis, the operator $\hat{L}_z$, $\hat{L}_z^2$, $\hat{S}$, and $\hat{S}^2$. Are these operators observables?
\item[(b)] Give the form of the most general matrix which represents an operator which commutes with $\hat{L}_z$. Same question for $\hat{L}_z^2$, then $\hat{S}^2$.
\item[(c)] Do $\hat{L}_z^2$ and $\hat{S}$ form a CSCO? Give a basis of common eigenvectors.
\end{itemize}
\end{problem}
\begin{sol}
\begin{itemize}
% \item[(a)] Diagonalize $L_z$
% \begin{gather}
% \left(\begin{array}{ccc}
% 1&0&0\\
% 0&\frac{1}{\sqrt{2}}&\frac{1}{\sqrt{2}}\\
% 0&\frac{1}{\sqrt{2}}&\frac{1}{\sqrt{2}}\\
% \end{array}\right)^{-1}\hat{L}_z\left(\begin{array}{ccc}
% 1&0&0\\
% 0&\frac{1}{\sqrt{2}}&\frac{1}{\sqrt{2}}\\
% 0&\frac{1}{\sqrt{2}}&\frac{1}{\sqrt{2}}\\
% \end{array}\right)=\left(\begin{array}{ccc}
% 1&0&0\\
% 0&0&0\\
% 0&0&-1\\
% \end{array}\right)\\
% \Longrightarrow\hat{L}_z=\left(\begin{array}{ccc}
% 1&0&0\\
% 0&\frac{1}{\sqrt{2}}&\frac{1}{\sqrt{2}}\\
% 0&\frac{1}{\sqrt{2}}&\frac{1}{\sqrt{2}}\\
% \end{array}\right)\left(\begin{array}{ccc}
% 1&0&0\\
% 0&0&0\\
% 0&0&-1\\
% \end{array}\right)\left(\begin{array}{ccc}
% 1&0&0\\
% 0&\frac{1}{\sqrt{2}}&\frac{1}{\sqrt{2}}\\
% 0&\frac{1}{\sqrt{2}}&\frac{1}{\sqrt{2}}\\
% \end{array}\right)^{-1}=\left(\begin{array}{ccc}
% 1&0&0\\
% 0&-\frac{1}{2}&\frac{1}{2}\\
% 0&\frac{1}{2}&-\frac{1}{2}\\
% \end{array}\right)
% \end{gather}
% so
% \begin{equation}
% \hat{L}_z^2=\left(\begin{array}{ccc}
% 1&0&0\\
% 0&-\frac{1}{2}&\frac{1}{2}\\
% 0&\frac{1}{2}&-\frac{1}{2}\\
% \end{array}\right)\left(\begin{array}{ccc}
% 1&0&0\\
% 0&-\frac{1}{2}&\frac{1}{2}\\
% 0&\frac{1}{2}&-\frac{1}{2}\\
% \end{array}\right)=\left(\begin{array}{ccc}
% 1&0&0\\
% 0&\frac{1}{2}&-\frac{1}{2}\\
% 0&-\frac{1}{2}&\frac{1}{2}\\
% \end{array}\right)
% \end{equation}
% Diagonalize $S$
% \begin{gather}
% \left(\begin{array}{ccc}
% 1&0&0\\
% 0&\frac{1}{\sqrt{2}}&\frac{1}{\sqrt{2}}\\
% 0&\frac{1}{\sqrt{2}}&\frac{1}{\sqrt{2}}\\
% \end{array}\right)^{-1}\hat{S}\left(\begin{array}{ccc}
% 1&0&0\\
% 0&\frac{1}{\sqrt{2}}&\frac{1}{\sqrt{2}}\\
% 0&\frac{1}{\sqrt{2}}&\frac{1}{\sqrt{2}}\\
% \end{array}\right)=\left(\begin{array}{ccc}
% 1&0&0\\
% 0&1&0\\
% 0&0&1\\
% \end{array}\right)\\
% \Longrightarrow\hat{S}=\left(\begin{array}{ccc}
% 1&0&0\\
% 0&\frac{1}{\sqrt{2}}&\frac{1}{\sqrt{2}}\\
% 0&\frac{1}{\sqrt{2}}&\frac{1}{\sqrt{2}}\\
% \end{array}\right)\left(\begin{array}{ccc}
% 1&0&0\\
% 0&1&0\\
% 0&0&1\\
% \end{array}\right)\left(\begin{array}{ccc}
% 1&0&0\\
% 0&\frac{1}{\sqrt{2}}&\frac{1}{\sqrt{2}}\\
% 0&\frac{1}{\sqrt{2}}&\frac{1}{\sqrt{2}}\\
% \end{array}\right)^{-1}=\left(\begin{array}{ccc}
% 1&0&0\\
% 0&1&0\\
% 0&0&1\\
% \end{array}\right)
% \end{gather}
% so
% \begin{equation}
% \hat{S}^2=\left(\begin{array}{ccc}
% 1&0&0\\
% 0&1&0\\
% 0&0&1\\
% \end{array}\right)\left(\begin{array}{ccc}
% 1&0&0\\
% 0&1&0\\
% 0&0&1\\
% \end{array}\right)=\left(\begin{array}{ccc}
% 1&0&0\\
% 0&1&0\\
% 0&0&1\\
% \end{array}\right)
% \end{equation}
% \begin{equation}
% \hat{L}_z^{\dagger}=(\hat{L}_z^T)^*=\left(\begin{array}{ccc}
% 1&0&0\\
% 0&-\frac{1}{2}&\frac{1}{2}\\
% 0&\frac{1}{2}&-\frac{1}{2}\\
% \end{array}\right)=\hat{L}_z
% \end{equation}
% so $\hat{L}_z$ is Hermitian.
% \begin{equation}
% (\hat{L}_z^2)^{\dagger}=[(\hat{L}_z^2)^T]^*=\left(\begin{array}{ccc}
% 1&0&0\\
% 0&\frac{1}{2}&-\frac{1}{2}\\
% 0&-\frac{1}{2}&\frac{1}{2}\\
% \end{array}\right)=\hat{L}_z^2
% \end{equation}
% so $\hat{L}_z^2$ is Hermitian.
% \begin{equation}
% \hat{S}^{\dagger}=(\hat{S}^T)^*=\left(\begin{array}{ccc}
% 1&0&0\\
% 0&1&0\\
% 0&0&1\\
% \end{array}\right)=\hat{S}
% \end{equation}
% so $\hat{S}$ is Hermitian.
% \begin{equation}
% (\hat{S}^2)^{\dagger}=[(\hat{S}^2)^T]^*=\left(\begin{array}{ccc}
% 1&0&0\\
% 0&1&0\\
% 0&0&1\\
% \end{array}\right)=\hat{S}^2
% \end{equation}
% so $\hat{S}^2$ is Hermitian.
% \begin{align}
% \nonumber\sum_i|u_i\rangle\langle u_i|=&\left(\begin{array}{c}
% 1\\
% 0\\
% 0\\
% \end{array}\right)\left(\begin{array}{ccc}
% 1&0&0\\
% \end{array}\right)+\left(\begin{array}{c}
% 0\\
% \frac{1}{\sqrt{2}}\\
% \frac{1}{\sqrt{2}}\\
% \end{array}\right)\left(\begin{array}{ccc}
% 0&\frac{1}{\sqrt{2}}&\frac{1}{\sqrt{2}}\\
% \end{array}\right)\\
% \nonumber&+\left(\begin{array}{c}
% 0\\
% \frac{1}{\sqrt{2}}\\
% -\frac{1}{\sqrt{2}}\\
% \end{array}\right)\left(\begin{array}{ccc}
% 0&\frac{1}{\sqrt{2}}&-\frac{1}{\sqrt{2}}\\
% \end{array}\right)\\
% =&\left(\begin{array}{ccc}
% 1&0&0\\
% 0&1&0\\
% 0&0&1\\
% \end{array}\right)=I
% \end{align}
% so their orthonormal system of eigenvectors forms basis in the state space.

% Therefore, these operators are observables.
\item[(a)] Diagonalize $L_z$
\begin{gather}
\left(\begin{array}{ccc}
1&0&0\\
0&1&0\\
0&0&1\\
\end{array}\right)^{-1}\hat{L}_z\left(\begin{array}{ccc}
1&0&0\\
0&1&0\\
0&0&1\\
\end{array}\right)=\left(\begin{array}{ccc}
1&0&0\\
0&0&0\\
0&0&-1\\
\end{array}\right)\\
\Longrightarrow\hat{L}_z=\left(\begin{array}{ccc}
1&0&0\\
0&1&0\\
0&0&1\\
\end{array}\right)\left(\begin{array}{ccc}
1&0&0\\
0&0&0\\
0&0&-1\\
\end{array}\right)\left(\begin{array}{ccc}
1&0&0\\
0&1&0\\
0&0&1\\
\end{array}\right)^{-1}=\left(\begin{array}{ccc}
1&0&0\\
0&0&0\\
0&0&-1\\
\end{array}\right)
\end{gather}
so
\begin{equation}
\hat{L}_z^2=\left(\begin{array}{ccc}
1&0&0\\
0&0&0\\
0&0&-1\\
\end{array}\right)\left(\begin{array}{ccc}
1&0&0\\
0&0&0\\
0&0&-1\\
\end{array}\right)=\left(\begin{array}{ccc}
1&0&0\\
0&0&0\\
0&0&1\\
\end{array}\right)
\end{equation}
Diagonalize $S$
\begin{gather}
\left(\begin{array}{ccc}
1&0&0\\
0&1&0\\
0&0&1\\
\end{array}\right)^{-1}\hat{S}\left(\begin{array}{ccc}
1&0&0\\
0&1&0\\
0&0&1\\
\end{array}\right)=\left(\begin{array}{ccc}
0&0&1\\
0&1&0\\
1&0&0\\
\end{array}\right)\\
\Longrightarrow\hat{S}=\left(\begin{array}{ccc}
1&0&0\\
0&1&0\\
0&0&1\\
\end{array}\right)\left(\begin{array}{ccc}
0&0&1\\
0&1&0\\
1&0&0\\
\end{array}\right)\left(\begin{array}{ccc}
1&0&0\\
0&1&0\\
0&0&1\\
\end{array}\right)^{-1}=\left(\begin{array}{ccc}
0&0&1\\
0&1&0\\
1&0&0\\
\end{array}\right)
\end{gather}
so
\begin{equation}
\hat{S}^2=\left(\begin{array}{ccc}
0&0&1\\
0&1&0\\
1&0&0\\
\end{array}\right)\left(\begin{array}{ccc}
0&0&1\\
0&1&0\\
1&0&0\\
\end{array}\right)=\left(\begin{array}{ccc}
1&0&0\\
0&1&0\\
0&0&1\\
\end{array}\right)
\end{equation}
\begin{equation}
\hat{L}_z^{\dagger}=(\hat{L}_z^T)^*=\left(\begin{array}{ccc}
1&0&0\\
0&0&0\\
0&0&-1\\
\end{array}\right)=\hat{L}_z
\end{equation}
so $\hat{L}_z$ is Hermitian.
\begin{equation}
(\hat{L}_z^2)^{\dagger}=[(\hat{L}_z^2)^T]^*=\left(\begin{array}{ccc}
1&0&0\\
0&0&0\\
0&0&1\\
\end{array}\right)=\hat{L}_z^2
\end{equation}
so $\hat{L}_z^2$ is Hermitian.
\begin{equation}
\hat{S}^{\dagger}=(\hat{S}^T)^*=\left(\begin{array}{ccc}
0&0&1\\
0&1&0\\
1&0&0\\
\end{array}\right)=\hat{S}
\end{equation}
so $\hat{S}$ is Hermitian.
\begin{equation}
(\hat{S}^2)^{\dagger}=[(\hat{S}^2)^T]^*=\left(\begin{array}{ccc}
1&0&0\\
0&1&0\\
0&0&1\\
\end{array}\right)=\hat{S}^2
\end{equation}
so $\hat{S}^2$ is Hermitian.
\begin{align}
\nonumber\sum_i|u_i\rangle\langle u_i|=&\left(\begin{array}{c}
1\\
0\\
0\\
\end{array}\right)\left(\begin{array}{ccc}
1&0&0\\
\end{array}\right)+\left(\begin{array}{c}
0\\
1\\
0\\
\end{array}\right)\left(\begin{array}{ccc}
0&1&0\\
\end{array}\right)\\
\nonumber&+\left(\begin{array}{c}
0\\
0\\
1\\
\end{array}\right)\left(\begin{array}{ccc}
0&0&1\\
\end{array}\right)\\
=&\left(\begin{array}{ccc}
1&0&0\\
0&1&0\\
0&0&1\\
\end{array}\right)=I
\end{align}
so their orthonormal system of eigenvectors forms basis in the state space.\\
Therefore, these operators are observables.
\item[(b)] Suppose the most general matrix
\begin{equation}
A=\left(\begin{array}{ccc}
a_{11}&a_{12}&a_{13}\\
a_{21}&a_{22}&a_{23}\\
a_{31}&a_{32}&a_{33}\\
\end{array}\right)
\end{equation}
If $A$ commute with $\hat{L}_z$
\begin{gather}
\begin{align}
\nonumber[A,\hat{L}_z]&=A\hat{L}_z-\hat{L}_zA\\
\nonumber=&\left(\begin{array}{ccc}
a_{11}&a_{12}&a_{13}\\
a_{21}&a_{22}&a_{23}\\
a_{31}&a_{32}&a_{33}\\
\end{array}\right)\left(\begin{array}{ccc}
1&0&0\\
0&0&0\\
0&0&-1\\
\end{array}\right)-\left(\begin{array}{ccc}
1&0&0\\
0&0&0\\
0&0&-1\\
\end{array}\right)\left(\begin{array}{ccc}
a_{11}&a_{12}&a_{13}\\
a_{21}&a_{22}&a_{23}\\
a_{31}&a_{32}&a_{33}\\
\end{array}\right)\\
=&\left(\begin{array}{ccc}
0&-a_{12}&-2a_{12}\\
a_{21}&0&-a_{23}\\
2a_{31}&a_{32}&0\\
\end{array}\right)=\left(\begin{array}{ccc}
0&0&0\\
0&0&0\\
0&0&0\\
\end{array}\right)
\end{align}\\
\Longrightarrow a_{12}=a_{13}=a_{21}=a_{23}=a_{31}=a_{32}=0
\end{gather}
so the most general matrix commuting with $\hat{L}_z$ is
\begin{equation}
\left(\begin{array}{ccc}
a_{11}&0&0\\
0&a_{22}&0\\
0&0&a_{33}\\
\end{array}\right)
\end{equation}
If $A$ commute with $\hat{L}_z^2$
\begin{gather}
\begin{align}
\nonumber[A,\hat{L}_z^2]&=A\hat{L}_z^2-\hat{L}_z^2A\\
\nonumber=&\left(\begin{array}{ccc}
a_{11}&a_{12}&a_{13}\\
a_{21}&a_{22}&a_{23}\\
a_{31}&a_{32}&a_{33}\\
\end{array}\right)\left(\begin{array}{ccc}
1&0&0\\
0&0&0\\
0&0&1\\
\end{array}\right)-\left(\begin{array}{ccc}
1&0&0\\
0&0&0\\
0&0&-1\\
\end{array}\right)\left(\begin{array}{ccc}
a_{11}&a_{12}&a_{13}\\
a_{21}&a_{22}&a_{23}\\
a_{31}&a_{32}&a_{33}\\
\end{array}\right)\\
=&\left(\begin{array}{ccc}
0&a_{12}&0\\
a_{21}&0&a_{23}\\
0&a_{32}&0\\
\end{array}\right)=\left(\begin{array}{ccc}
0&0&0\\
0&0&0\\
0&0&0\\
\end{array}\right)
\end{align}\\
\Longrightarrow a_{12}=a_{21}=a_{23}=a_{32}=0
\end{gather}
so the most general matrix commuting with $\hat{L}_z^2$ is
\begin{equation}
\left(\begin{array}{ccc}
a_{11}&0&a_{13}\\
0&a_{22}&0\\
a_{31}&0&a_{33}\\
\end{array}\right)
\end{equation}
If $A$ commute with $\hat{S}$
\begin{gather}
\begin{align}
\nonumber[A,\hat{S}]&=A\hat{S}-\hat{S}A\\
\nonumber=&\left(\begin{array}{ccc}
a_{11}&a_{12}&a_{13}\\
a_{21}&a_{22}&a_{23}\\
a_{31}&a_{32}&a_{33}\\
\end{array}\right)\left(\begin{array}{ccc}
0&0&1\\
0&1&0\\
1&0&0\\
\end{array}\right)-\left(\begin{array}{ccc}
0&0&1\\
0&1&0\\
1&0&0\\
\end{array}\right)\left(\begin{array}{ccc}
a_{11}&a_{12}&a_{13}\\
a_{21}&a_{22}&a_{23}\\
a_{31}&a_{32}&a_{33}\\
\end{array}\right)\\
=&\left(\begin{array}{ccc}
0&0&0\\
0&0&0\\
0&0&0\\
\end{array}\right)
\end{align}\\
\Longrightarrow a_{11}=a_{12}=a_{13}=a_{21}=a_{22}=a_{23}=a_{31}=a_{32}=a_{33}=0
\end{gather}
so the most general matrix commuting with $\hat{S}$ is
\begin{equation}
\left(\begin{array}{ccc}
a_{11}&a_{12}&a_{13}\\
a_{21}&a_{22}&a_{23}\\
a_{31}&a_{32}&a_{33}\\
\end{array}\right)
\end{equation}
If $A$ commute with $\hat{S}^2$
\begin{gather}
\begin{align}
\nonumber[A,\hat{S}^2]&=A\hat{S}^2-\hat{S}^2A\\
\nonumber=&\left(\begin{array}{ccc}
a_{11}&a_{12}&a_{13}\\
a_{21}&a_{22}&a_{23}\\
a_{31}&a_{32}&a_{33}\\
\end{array}\right)\left(\begin{array}{ccc}
1&0&0\\
0&1&0\\
0&0&1\\
\end{array}\right)-\left(\begin{array}{ccc}
1&0&0\\
0&1&0\\
0&0&1\\
\end{array}\right)\left(\begin{array}{ccc}
a_{11}&a_{12}&a_{13}\\
a_{21}&a_{22}&a_{23}\\
a_{31}&a_{32}&a_{33}\\
\end{array}\right)\\
=&\left(\begin{array}{ccc}
0&0&0\\
0&0&0\\
0&0&0\\
\end{array}\right)
\end{align}\\
\Longrightarrow a_{11}=a_{12}=a_{13}=a_{21}=a_{22}=a_{23}=a_{31}=a_{32}=a_{33}=0
\end{gather}
so the most general matrix commuting with $\hat{S}^2$ is
\begin{equation}
\left(\begin{array}{ccc}
a_{11}&a_{12}&a_{13}\\
a_{21}&a_{22}&a_{23}\\
a_{31}&a_{32}&a_{33}\\
\end{array}\right)
\end{equation}
\item[(c)]
\begin{align}
\nonumber[\hat{L}_z^2,\hat{S}]=&\hat{L}_z^2\hat{S}-\hat{S}\hat{L}_z^2=\left(\begin{array}{ccc}
1&0&0\\
0&0&0\\
0&0&1\\
\end{array}\right)\left(\begin{array}{ccc}
0&0&1\\
0&1&0\\
1&0&0\\
\end{array}\right)-\left(\begin{array}{ccc}
0&0&1\\
0&1&0\\
1&0&0\\
\end{array}\right)\left(\begin{array}{ccc}
1&0&0\\
0&0&0\\
0&0&1\\
\end{array}\right)\\
=&\left(\begin{array}{ccc}
0&0&0\\
0&0&0\\
0&0&0\\
\end{array}\right)
\end{align}
so $\hat{L}_z^2$ and $\hat{S}$ commute.\\
$|u_2\rangle=\left(\begin{array}{c}0\\1\\0\end{array}\right)$ is a common eigenvector of $\hat{L}_z^2$ and $\hat{S}$.
\begin{equation}
\hat{L}_z^2|u_2\rangle=0|u_2\rangle,\quad\hat{S}|u_2\rangle=1|u_2\rangle
\end{equation}
 In the subspace spanned by $|u_1\rangle=\left(\begin{array}{c}1\\0\\0\end{array}\right)$ and $|u_3\rangle=\left(\begin{array}{c}0\\0\\1\end{array}\right)$,
\begin{gather}
\sum_{i=1,3}|u_i\rangle\langle u_i|\hat{L}_z^2\sum_{i=1,3}|u_i\rangle\langle u_i|=\left(\begin{array}{ccc}
1&0&0\\
0&0&0\\
0&0&1\\
\end{array}\right)\left(\begin{array}{ccc}
1&0&0\\
0&0&0\\
0&0&1\\
\end{array}\right)\left(\begin{array}{ccc}
0&0&1\\
0&1&0\\
1&0&0\\
\end{array}\right)
=\left(\begin{array}{ccc}
1&0&0\\
0&0&0\\
0&0&1\\
\end{array}\right)\\
\sum_{i=1,3}|u_i\rangle\langle u_i|\hat{S}\sum_{i=1,3}|u_i\rangle\langle u_i|=\left(\begin{array}{ccc}
1&0&0\\
0&0&0\\
0&0&1\\
\end{array}\right)\left(\begin{array}{ccc}
0&0&1\\
0&1&0\\
1&0&0\\
\end{array}\right)\left(\begin{array}{ccc}
1&0&0\\
0&0&0\\
0&0&1\\
\end{array}\right)
=\left(\begin{array}{ccc}
0&0&1\\
0&0&0\\
1&0&0\\
\end{array}\right)
\end{gather}
The other two common eigenvectors are
\begin{gather}
|\psi_2\rangle=\left(\begin{array}{c}
\frac{1}{\sqrt{2}}\\
0\\
\frac{1}{\sqrt{2}}\\
\end{array}\right)\\
|\psi_3\rangle=\left(\begin{array}{c}
\frac{1}{\sqrt{2}}\\
0\\
-\frac{1}{\sqrt{2}}\\
\end{array}\right)
\end{gather}
\begin{gather}
\hat{L}_z^2|\psi_2\rangle=1|\psi_2\rangle\\
\hat{L}_z^2|\psi_3\rangle=1|\psi_3\rangle\\
\hat{S}|\psi_2\rangle=1|\psi_2\rangle\\
\hat{S}|\psi_3\rangle=-1|\psi_3\rangle
\end{gather}
so specifying the eigenvalues of $\hat{L}_z^2$ and $\hat{S}$ determine a unique set of common eigenvector $\left\{\left(\begin{array}{c}0\\1\\0\\\end{array}\right),\left(\begin{array}{c}\frac{1}{\sqrt{2}}\\0\\\frac{1}{\sqrt{2}}\\\end{array}\right),\left(\begin{array}{c}\frac{1}{\sqrt{2}}\\0\\-\frac{1}{\sqrt{2}}\\\end{array}\right)\right\}$.\\
Therefore, $\hat{L}_z^2$ and $\hat{S}$ form a CSCO.
\end{itemize}
\end{sol}
\end{document}